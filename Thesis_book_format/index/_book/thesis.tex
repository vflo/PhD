% This is the Reed College LaTeX thesis template. Most of the work
% for the document class was done by Sam Noble (SN), as well as this
% template. Later comments etc. by Ben Salzberg (BTS). Additional
% restructuring and APA support by Jess Youngberg (JY).
% Your comments and suggestions are more than welcome; please email
% them to cus@reed.edu
%
% See http://web.reed.edu/cis/help/latex.html for help. There are a
% great bunch of help pages there, with notes on
% getting started, bibtex, etc. Go there and read it if you're not
% already familiar with LaTeX.
%
% Any line that starts with a percent symbol is a comment.
% They won't show up in the document, and are useful for notes
% to yourself and explaining commands.
% Commenting also removes a line from the document;
% very handy for troubleshooting problems. -BTS

% As far as I know, this follows the requirements laid out in
% the 2002-2003 Senior Handbook. Ask a librarian to check the
% document before binding. -SN

%%
%% Preamble
%%
% \documentclass{<something>} must begin each LaTeX document
\documentclass[11pt,twoside]{reedthesis}
\usepackage[paperheight=237mm,paperwidth=168mm,bottom=20mm, top=20mm, outer=1.5cm, inner=2.4cm]{geometry}
% Packages are extensions to the basic LaTeX functions. Whatever you
% want to typeset, there is probably a package out there for it.
% Chemistry (chemtex), screenplays, you name it.
% Check out CTAN to see: http://www.ctan.org/
%%
\usepackage{graphicx,latexsym}
\usepackage{amsmath}
\usepackage{amssymb,amsthm}
\usepackage{longtable,booktabs,setspace}
\usepackage{chemarr} %% Useful for one reaction arrow, useless if you're not a chem major
\usepackage[hyphens]{url}
% Added by CII
\usepackage{hyperref}
\usepackage{lmodern}
\usepackage{float}
\floatplacement{figure}{H}
% End of CII addition
\usepackage{rotating}

% Next line commented out by CII
%%% \usepackage{natbib}
% Comment out the natbib line above and uncomment the following two lines to use the new
% biblatex-chicago style, for Chicago A. Also make some changes at the end where the
% bibliography is included.
%\usepackage{biblatex-chicago}
%\bibliography{thesis}


% Added by CII (Thanks, Hadley!)
% Use ref for internal links
\renewcommand{\hyperref}[2][???]{\autoref{#1}}
\def\chapterautorefname{Chapter}
\def\sectionautorefname{Section}
\def\subsectionautorefname{Subsection}
% End of CII addition

% Added by CII
\usepackage{caption}
\captionsetup{width=5in}
% End of CII addition

% \usepackage{times} % other fonts are available like times, bookman, charter, palatino

% Syntax highlighting #22

% To pass between YAML and LaTeX the dollar signs are added by CII
\title{Global ecological drivers of transpiration regulation in woody plants}
\author{Víctor Flo Sierra}
% The month and year that you submit your FINAL draft TO THE LIBRARY (May or December)
\date{Dec 2020}
\degree{DOCTORADO EN ECOLOGIA TERRESTRE}
% \division{}
\advisor{Rafael Poyatos López}
\institution{Autonomous University of Barcelona}
% \degree{DOCTORADO EN ECOLOGIA TERRESTRE}
%If you have two advisors for some reason, you can use the following
% Uncommented out by CII
\altadvisor{Jordi Martínez Vilalta}
% End of CII addition

%%% Remember to use the correct department!
\department{Centre de Recerca Ecològica i Aplicacions Forestals}
% if you're writing a thesis in an interdisciplinary major,
% uncomment the line below and change the text as appropriate.
% check the Senior Handbook if unsure.
%\thedivisionof{The Established Interdisciplinary Committee for}
% if you want the approval page to say "Approved for the Committee",
% uncomment the next line
%\approvedforthe{Committee}

% Added by CII
%%% Copied from knitr
%% maxwidth is the original width if it's less than linewidth
%% otherwise use linewidth (to make sure the graphics do not exceed the margin)
\makeatletter
\def\maxwidth{ %
  \ifdim\Gin@nat@width>\linewidth
    \linewidth
  \else
    \Gin@nat@width
  \fi
}
\makeatother

\renewcommand{\contentsname}{Table of Contents}
% End of CII addition

\setlength{\parskip}{0pt}

% Added by CII

\providecommand{\tightlist}{%
  \setlength{\itemsep}{0pt}\setlength{\parskip}{0pt}}

\Acknowledgements{
\setlength{\parindent}{30pt} A pesar de ser firme defensora de que la
tesis no supone un mérito excepcional digno de especiales
reconocimientos, no desperdiciaré la oportunidad de agradecer todo lo
vivido durante estos años y dejar grabado lo mucho que deben estas
páginas a las personas que me rodean.\par
Durante esta etapa, que empezó cuando llegué a Barcelona, he tenido la
suerte de encontrar a personas maravillosas, de viajar por el mundo y de
aprender sobre diferentes aspectos de la vida. Quiero agradecer en
primer lugar, a la que fue mi primera familia en Catalunya, mis
compañeros de máster, en especial a Estrella, Aida, Alba, Aina, Carla,
Enrique, John y Gustavo y a mis compañeros de piso Laura y Josan.
Gracias por los ratos en el césped, las visitas a Mataró y las noches
arreglando el mundo en el \emph{Cop de ma}, gracias, en definitiva, por
hacerme sentir en casa.\par
Cuando una empieza la tesis le dicen que debería preocuparse de escoger
bien a sus directores porque son personas con las que tendrá que
compartir cuatro años y ver más a menudo que a sus propios padres. Yo
con Paco apenas tuve dos reuniones antes de solicitar el doctorado.
Después de estos más de cuatro años trabajando juntos sé que si volviese
atrás volvería a elegirte. Gracias Paco por el ejemplo, por las charlas
en los viajes, por tu forma de pensar tan crítica, transversal y
profunda, por enseñarme tanto más allá de lo estrictamente
académico\ldots{} A Miguel Ángel lo conocía más. Fui su alumna durante
la carrera y sé que pocas personas tienen un conocimiento más holístico
de la naturaleza murciana, mayor integridad y devoción por su trabajo.
Gracias por tu visión naturalista y aplicada, gracias por las charlas de
política y las explicaciones sobre el Mar Menor. Durante estos años he
crecido con y gracias a vosotros, y no podría sentirme más
afortunada.\par
Además esta tesis no hubiese tenido los mismos resultados sino hubiese
compartido las dudas, preocupaciones, alegrías y el desánimo con tantos
buenos compañeros. Gracias a los compañeros de despacho, a Anna, Javi,
Judit, Carlos, Manu, Marta, Pere y Pol. Sin duda las penas han sido
menos penas con vosotros y las fiestas, almuerzos y barbacoas mucho más
divertidas. Gracias por las risas y la terapia de grupo. No me
acostumbro a pensar en un trabajo sin vosotros.\par
Gracias también a los compañeros de grupo. Enric, Jordi, Luciana y
Nuria, el \emph{phoskitos lab} no podría tener mejor plantilla. Con
vosotros se demuestra que hacer ciencia no está reñido con un ambiente
laboral de calidad, colaborativo, respetuoso y divertido. Gracias por
tantos buenos momentos, por las discusiones académicas y las no tan
académicas, por las salidas a la montaña y las jornadas de convivencia
en los congresos. Quiero agradecer también a mis compañeros de
departamento en Murcia. Gracias a Paqui, Juan Miguel y Pablo por el
trabajo de campo y de despacho, por estar siempre ahí pese a la
distancia, por no haber dudado en ayudar siempre que lo he necesitado.
Gracias también a Guillem por las semanas de trabajo y convivencia en
Murcia y tus dosis de amabilidad y optimismo, a Pep Serra y Gerard Sapes
por los sabios consejos y la inspiración.\par
Durante estos últimos años también he tenido la oportunidad de librarme
de los calurosos veranos catalanes y murcianos escapándome de estancia a
otros países. Gracias a Jens-Christian por la acogida en Aarhus y a
Antoine y Olivier por la estancia en Suiza. Me quedo con el recuerdo de
los paseos en bici por Dinamarca y la estampa de los Alpes por la
ventana del despacho y las barbacoas en el lago Leman. Gracias también a
los que han estado siempre ahí, a los amigos de toda la vida, Jose,
Maria José, Mari Nieves, Maria Victoria, Álvaro Talavera y Pedro.
Gracias por estar tan cerca a pesar de los kilómetros, gracias por las
visitas, por aguantar con paciencia mis quejas sobre política y ciencia,
por los cafés, las terapias y el consuelo. Gracias también a Victor
Valero por el tiempo que hemos pasado juntos en Barcelona, por las horas
de desahogo mutuo y los planes culturales.\par
Quiero agradecer también a mi familia. A mis tios, por ayudarme siempre
que les ha sido posible. A mi hermano mayor, por plantar en mi la
semilla del amor por la naturaleza, sin tu ejemplo no hubiese dado los
primeros pasos del camino que ahora completo. Gracias también a mi
hermano mellizo por la ayuda incondicional e implicación, llegando
incluso a armarte con botas y cinta métrica para echar una mano en el
trabajo de campo.\par
Gracias sobre todo a quien desde hace más de tres años es mucho más que
un compañero de despacho. Gracias Víctor por tu compresión, tu
paciencia, tus consejos y buenas ideas. Estas páginas te deben mucho.
Gracias por animarme y hacerme ver siempre el vaso medio lleno. Por
darme una segunda familia en Catalunya.\par
Gracias finalmente a mis padres, a quienes una época cruel y la falta de
recursos les despojó de la oportunidad para completar etapas académicas
más allá de los estudios más elementales. Sabed que no habría sido
posible llegar hasta aquí sin vosotros.\par
}

\Dedication{
\vspace*{4.5cm}
\begin{flushright}
\hfil \textit{Lo que nos diferencia de otros seres vivos} \break
\hfil \textit{no es la capacidad de perturbar el planeta,} \break
\hfil \textit{sino la voluntad de salvarlo.}
\end{flushright}
\vspace*{\fill}
}

\Preface{

}

\Abstract{
\setlength{\parindent}{30pt} Understanding how climate affects species'
distribution and performance is a central issue in ecology since its
origins. In last decades, however, the interest in this question has
been reactivated by the current context of climate change. Species Niche
Modelling has been widely used to assess shifts in species distribution
and to test the relationship between species' climatic niche and species
physiological and demographic performance, implicitly assuming that
species occurrence portrays the environmental and biotic species'
suitable conditions. Nevertheless it is still largely undetermined
whether these models can portray population and community responses,
particularly in relation to extreme climatic episodes.\par

In this thesis I aim at exploring the capacity of niche modelling to
predict species decay under extreme climatic conditions, particularly
droughts, addressing some constraints of this approach and proposing
possible solutions. To achieve this goal, I counted with 3 vegetation
decay datasets measured in the Spanish SE after the extreme drought year
2013-2014. Two of these datasets were based on defoliation sampling of
individual plants belonging to more than 40 semiarid shrubland species
(chapters 2, 4 and 5), while the other one was based on regional
compiled data of \emph{Pinus halepensis} L. affectation in plots of
\(1 km^2\) (chapter 3). In second chapter I used different Species
Distribution Model (SDMs) algorithms to estimate species' climatic
suitability before (1950-2000) and during the extreme drought, in order
to test the possible correlation between suitability and decay, and
whether the existence of this relationship depended on the applied SDM
algorithm. I consistently found a positive correlation between remaining
green canopy and species' climatic suitability before the event,
suggesting that populations historically living closer to their species'
tolerance limits are more vulnerable to drought. Contrastingly,
decreased climatic suitability during the drought period did not
correlate with remaining green canopy, likely because of extremely low
climatic suitability values achieved during the exceptional climatic
episode. In order to test whether this extremely low suitability values
could derive as a consequence of only considering climatic averages when
calibrating SDMs, in the third chapter I developed a method to include
inter-annual climatic variability into niche characterization. I then
compared the respective capacities of climatic suitabilities obtained
from averaged-based and from inter-annual variability-based niches to
explain demographic responses to extreme climatic events. I found that
climatic suitability obtained from both niches quantifications
significantly explained species demographic responses. However, climatic
suitability from inter-annual variability-based niches showed higher
explanatory capacity, especially for populations that tend to be more
geographically marginal. In the fourth chapter I tried to overcome the
inability of the SDMs to predict populations decay during extreme
conditions, as observed in the second chapter, by using Euclidean
distances to species' niche in the environmental space. I compared the
capacities of both population distances in the climatic environmental
space and population climatic suitability derived from SDMs to explain
population observed physiological and demographic responses to an
extreme event. Additionally, I tested such relationship in populations
located in three different bedrock sites, corresponding to a gradient of
water availability. I found that SDMs-derived suitability failed to
explain population decay while distances to the niche centroid and limit
significantly explained population die-off, highlighting that population
displaced farther from species' niche during the extreme episode showed
higher vulnerability to drought. The results also suggested a relevant
role of some bedrocks buffering species decay responses to extreme
drought events mainly according to soil water holding capacity. Finally,
in the fifth chapter, I used species niche characterizations in the
environmental space and demographic data to address the impact of
extreme events at community level. Particularly, I estimated the
community climatic disequilibrium before and after a drought episode
along a gradient of water availability in three bedrock types.
Disequilibrium was computed as the difference between observed climate
and community-inferred climate, which was calculated as the mean of
species' climatic optimum weighted by species abundance collected in
field surveys. I found that extreme drought nested within a decadal
trend of increasingly aridity led to a reduction in community climatic
disequilibrium, particularly when combined with low water-retention
bedrocks. In addition, community climatic disequilibrium also varied
before the extreme event across bedrock types, according to soils
water-retention capacity. In conclusion, by developing different
techniques, derived from species distribution, that characterize
climatic accuracy at population and community level, this work reveals
the capacity of species climatic niche to explain demographic responses
under climate change-induced episodes of extreme drought.\par
}

	\usepackage{tikz}
\usepackage{parskip}
\usepackage{colortbl}
\usepackage{xcolor}
\usepackage{threeparttable}
\usepackage{pdflscape}
\usepackage{lettrine}
\usepackage{caption}
\usepackage{titlesec}
\usepackage{multirow}
\usepackage{crop}
\usepackage{pdfpages}
\usepackage{caption}
\usepackage{subcaption}
	\usepackage{booktabs}
\usepackage{longtable}
\usepackage{array}
\usepackage{multirow}
\usepackage{wrapfig}
\usepackage{float}
\usepackage{colortbl}
\usepackage{pdflscape}
\usepackage{tabu}
\usepackage{threeparttable}
\usepackage{threeparttablex}
\usepackage[normalem]{ulem}
\usepackage{makecell}
\usepackage{xcolor}
% End of CII addition
%%
%% End Preamble
%%
%
\begin{document}

% Everything below added by CII
  \maketitle

\frontmatter % this stuff will be roman-numbered
\pagestyle{empty} % this removes page numbers from the frontmatter
  \begin{acknowledgements}
    \setlength{\parindent}{30pt} A pesar de ser firme defensora de que la
    tesis no supone un mérito excepcional digno de especiales
    reconocimientos, no desperdiciaré la oportunidad de agradecer todo lo
    vivido durante estos años y dejar grabado lo mucho que deben estas
    páginas a las personas que me rodean.\par
    Durante esta etapa, que empezó cuando llegué a Barcelona, he tenido la
    suerte de encontrar a personas maravillosas, de viajar por el mundo y de
    aprender sobre diferentes aspectos de la vida. Quiero agradecer en
    primer lugar, a la que fue mi primera familia en Catalunya, mis
    compañeros de máster, en especial a Estrella, Aida, Alba, Aina, Carla,
    Enrique, John y Gustavo y a mis compañeros de piso Laura y Josan.
    Gracias por los ratos en el césped, las visitas a Mataró y las noches
    arreglando el mundo en el \emph{Cop de ma}, gracias, en definitiva, por
    hacerme sentir en casa.\par
    Cuando una empieza la tesis le dicen que debería preocuparse de escoger
    bien a sus directores porque son personas con las que tendrá que
    compartir cuatro años y ver más a menudo que a sus propios padres. Yo
    con Paco apenas tuve dos reuniones antes de solicitar el doctorado.
    Después de estos más de cuatro años trabajando juntos sé que si volviese
    atrás volvería a elegirte. Gracias Paco por el ejemplo, por las charlas
    en los viajes, por tu forma de pensar tan crítica, transversal y
    profunda, por enseñarme tanto más allá de lo estrictamente
    académico\ldots{} A Miguel Ángel lo conocía más. Fui su alumna durante
    la carrera y sé que pocas personas tienen un conocimiento más holístico
    de la naturaleza murciana, mayor integridad y devoción por su trabajo.
    Gracias por tu visión naturalista y aplicada, gracias por las charlas de
    política y las explicaciones sobre el Mar Menor. Durante estos años he
    crecido con y gracias a vosotros, y no podría sentirme más
    afortunada.\par
    Además esta tesis no hubiese tenido los mismos resultados sino hubiese
    compartido las dudas, preocupaciones, alegrías y el desánimo con tantos
    buenos compañeros. Gracias a los compañeros de despacho, a Anna, Javi,
    Judit, Carlos, Manu, Marta, Pere y Pol. Sin duda las penas han sido
    menos penas con vosotros y las fiestas, almuerzos y barbacoas mucho más
    divertidas. Gracias por las risas y la terapia de grupo. No me
    acostumbro a pensar en un trabajo sin vosotros.\par
    Gracias también a los compañeros de grupo. Enric, Jordi, Luciana y
    Nuria, el \emph{phoskitos lab} no podría tener mejor plantilla. Con
    vosotros se demuestra que hacer ciencia no está reñido con un ambiente
    laboral de calidad, colaborativo, respetuoso y divertido. Gracias por
    tantos buenos momentos, por las discusiones académicas y las no tan
    académicas, por las salidas a la montaña y las jornadas de convivencia
    en los congresos. Quiero agradecer también a mis compañeros de
    departamento en Murcia. Gracias a Paqui, Juan Miguel y Pablo por el
    trabajo de campo y de despacho, por estar siempre ahí pese a la
    distancia, por no haber dudado en ayudar siempre que lo he necesitado.
    Gracias también a Guillem por las semanas de trabajo y convivencia en
    Murcia y tus dosis de amabilidad y optimismo, a Pep Serra y Gerard Sapes
    por los sabios consejos y la inspiración.\par
    Durante estos últimos años también he tenido la oportunidad de librarme
    de los calurosos veranos catalanes y murcianos escapándome de estancia a
    otros países. Gracias a Jens-Christian por la acogida en Aarhus y a
    Antoine y Olivier por la estancia en Suiza. Me quedo con el recuerdo de
    los paseos en bici por Dinamarca y la estampa de los Alpes por la
    ventana del despacho y las barbacoas en el lago Leman. Gracias también a
    los que han estado siempre ahí, a los amigos de toda la vida, Jose,
    Maria José, Mari Nieves, Maria Victoria, Álvaro Talavera y Pedro.
    Gracias por estar tan cerca a pesar de los kilómetros, gracias por las
    visitas, por aguantar con paciencia mis quejas sobre política y ciencia,
    por los cafés, las terapias y el consuelo. Gracias también a Victor
    Valero por el tiempo que hemos pasado juntos en Barcelona, por las horas
    de desahogo mutuo y los planes culturales.\par
    Quiero agradecer también a mi familia. A mis tios, por ayudarme siempre
    que les ha sido posible. A mi hermano mayor, por plantar en mi la
    semilla del amor por la naturaleza, sin tu ejemplo no hubiese dado los
    primeros pasos del camino que ahora completo. Gracias también a mi
    hermano mellizo por la ayuda incondicional e implicación, llegando
    incluso a armarte con botas y cinta métrica para echar una mano en el
    trabajo de campo.\par
    Gracias sobre todo a quien desde hace más de tres años es mucho más que
    un compañero de despacho. Gracias Víctor por tu compresión, tu
    paciencia, tus consejos y buenas ideas. Estas páginas te deben mucho.
    Gracias por animarme y hacerme ver siempre el vaso medio lleno. Por
    darme una segunda familia en Catalunya.\par
    Gracias finalmente a mis padres, a quienes una época cruel y la falta de
    recursos les despojó de la oportunidad para completar etapas académicas
    más allá de los estudios más elementales. Sabed que no habría sido
    posible llegar hasta aquí sin vosotros.\par
  \end{acknowledgements}

  \hypersetup{linkcolor=black}
  \setcounter{tocdepth}{2}
  \tableofcontents

  \listoftables

  \listoffigures
  \begin{abstract}
    \setlength{\parindent}{30pt} Understanding how climate affects species'
    distribution and performance is a central issue in ecology since its
    origins. In last decades, however, the interest in this question has
    been reactivated by the current context of climate change. Species Niche
    Modelling has been widely used to assess shifts in species distribution
    and to test the relationship between species' climatic niche and species
    physiological and demographic performance, implicitly assuming that
    species occurrence portrays the environmental and biotic species'
    suitable conditions. Nevertheless it is still largely undetermined
    whether these models can portray population and community responses,
    particularly in relation to extreme climatic episodes.\par
    
    In this thesis I aim at exploring the capacity of niche modelling to
    predict species decay under extreme climatic conditions, particularly
    droughts, addressing some constraints of this approach and proposing
    possible solutions. To achieve this goal, I counted with 3 vegetation
    decay datasets measured in the Spanish SE after the extreme drought year
    2013-2014. Two of these datasets were based on defoliation sampling of
    individual plants belonging to more than 40 semiarid shrubland species
    (chapters 2, 4 and 5), while the other one was based on regional
    compiled data of \emph{Pinus halepensis} L. affectation in plots of
    \(1 km^2\) (chapter 3). In second chapter I used different Species
    Distribution Model (SDMs) algorithms to estimate species' climatic
    suitability before (1950-2000) and during the extreme drought, in order
    to test the possible correlation between suitability and decay, and
    whether the existence of this relationship depended on the applied SDM
    algorithm. I consistently found a positive correlation between remaining
    green canopy and species' climatic suitability before the event,
    suggesting that populations historically living closer to their species'
    tolerance limits are more vulnerable to drought. Contrastingly,
    decreased climatic suitability during the drought period did not
    correlate with remaining green canopy, likely because of extremely low
    climatic suitability values achieved during the exceptional climatic
    episode. In order to test whether this extremely low suitability values
    could derive as a consequence of only considering climatic averages when
    calibrating SDMs, in the third chapter I developed a method to include
    inter-annual climatic variability into niche characterization. I then
    compared the respective capacities of climatic suitabilities obtained
    from averaged-based and from inter-annual variability-based niches to
    explain demographic responses to extreme climatic events. I found that
    climatic suitability obtained from both niches quantifications
    significantly explained species demographic responses. However, climatic
    suitability from inter-annual variability-based niches showed higher
    explanatory capacity, especially for populations that tend to be more
    geographically marginal. In the fourth chapter I tried to overcome the
    inability of the SDMs to predict populations decay during extreme
    conditions, as observed in the second chapter, by using Euclidean
    distances to species' niche in the environmental space. I compared the
    capacities of both population distances in the climatic environmental
    space and population climatic suitability derived from SDMs to explain
    population observed physiological and demographic responses to an
    extreme event. Additionally, I tested such relationship in populations
    located in three different bedrock sites, corresponding to a gradient of
    water availability. I found that SDMs-derived suitability failed to
    explain population decay while distances to the niche centroid and limit
    significantly explained population die-off, highlighting that population
    displaced farther from species' niche during the extreme episode showed
    higher vulnerability to drought. The results also suggested a relevant
    role of some bedrocks buffering species decay responses to extreme
    drought events mainly according to soil water holding capacity. Finally,
    in the fifth chapter, I used species niche characterizations in the
    environmental space and demographic data to address the impact of
    extreme events at community level. Particularly, I estimated the
    community climatic disequilibrium before and after a drought episode
    along a gradient of water availability in three bedrock types.
    Disequilibrium was computed as the difference between observed climate
    and community-inferred climate, which was calculated as the mean of
    species' climatic optimum weighted by species abundance collected in
    field surveys. I found that extreme drought nested within a decadal
    trend of increasingly aridity led to a reduction in community climatic
    disequilibrium, particularly when combined with low water-retention
    bedrocks. In addition, community climatic disequilibrium also varied
    before the extreme event across bedrock types, according to soils
    water-retention capacity. In conclusion, by developing different
    techniques, derived from species distribution, that characterize
    climatic accuracy at population and community level, this work reveals
    the capacity of species climatic niche to explain demographic responses
    under climate change-induced episodes of extreme drought.\par
  \end{abstract}
  \begin{dedication}
    \vspace*{4.5cm}
    \begin{flushright}
    \hfil \textit{Lo que nos diferencia de otros seres vivos} \break
    \hfil \textit{no es la capacidad de perturbar el planeta,} \break
    \hfil \textit{sino la voluntad de salvarlo.}
    \end{flushright}
    \vspace*{\fill}
  \end{dedication}
\mainmatter % here the regular arabic numbering starts
\pagestyle{fancyplain} % turns page numbering back on

\chapter[Bias and uncertainty in sap flow methods]{A synthesis of bias and uncertainty in sap flow methods}

\setlength{\parindent}{0pt} Victor Flo, Jordi Martinez-Vilalta, Kathy
Steppe, Bernhard Schuldt, Rafael Poyatos \newpage
\setlength{\parindent}{30pt}

\section*{Abstract}

Sap flow measurements with thermometric methods are widely used to
measure transpiration in plants. Different method families exist
depending on how they apply heat and track sapwood temperature (heat
pulse, heat dissipation, heat field deformation or heat balance). These
methods have been calibrated for many species, but a global assessment
of their uncertainty and reliability has not yet been conducted. Here we
perform a meta-analysis of 290 individual calibration experiments
assembled from the literature to assess calibration performance and how
this varies across methods, experimental conditions and wood properties
(density and porosity types). We used different metrics to characterize
mean accuracy (closeness of the measurements to the true, reference
value), proportional bias (resulting from an effect of measured flow on
the magnitude of the error), linearity in the relationship between
measurements and reference values, and precision (reproducibility and
repeatability). We found a large intra- and inter-method variability in
calibration performance, with a low proportion of this variability
explained by species. Calibration performance was best when using stem
segments. We did not find evidence of strong effects of wood density or
porosity type in calibration performance. Dissipation methods showed
lower accuracy and higher proportional bias than the other methods but
they showed relatively high linearity and precision. Pulse methods also
showed significant proportional bias, driven by their overestimation of
low flows. These results suggest that Dissipation methods may be more
appropriate to assess relative sap flow (e.g., treatment effects within
a study) and Pulse methods may be more suitable to quantify absolute
flows. Nevertheless, all sap flow methods showed high precision,
allowing potential correction of the measurements when a study-specific
calibration is performed. Our understanding of how sap flow methods
perform across species would be greatly improved if experimental
conditions and wood properties, including changes in wood moisture, were
better reported.\par

\newpage

\section{Introduction}\label{introduction}

Quantifying transpiration of vegetation is of major importance for
hydrological, ecological, and agricultural sciences, since it represents
60-80\% of the water that returns from the land surface to the
atmosphere (Jasechko et al. (2013); Schlesinger \& Jasechko (2014); Wei
et al. (2017)). The study of transpiration and its environmental
sensitivity is essential to understand vegetation water cycling (D. C.
Frank et al. (2015); Konings, Williams, \& Gentine (2017); Novick et al.
(2016)) and to forecast changes in vegetation functioning and
composition under climate change (C. D. Allen, Breshears, \& McDowell
(2015)). Addressing these questions requires non-destructive
measurements of whole-plant transpiration at multiple timescales (S. D.
Wullschleger, Meinzer, \& Vertessy (1998)). Thermal methods of sap flow
measurement show a number of advantages over other methods such as those
based on isotopes tracing or leaf gas exchange (D. M. Smith (1995)), and
have become the most widely used approach to estimate tree-level
transpiration (Poyatos et al. (2016)) (Fig. A1). When compared against
independent estimates of evapotranspiration components, sap flow methods
have provided reasonable qualitative and quantitative results (Diawara,
Loustau, \& Berbigier (1991), Hogg et al. (1997), Kool et al. (2014),
Schlesinger \& Jasechko (2014), Zhang, Manzoni, Katul, Porporato, \&
Yang (2014), but see K. B. Wilson, Hanson, Mulholland, Baldocchi, \&
Wullschleger (2001), A. C. Oishi, Oren, \& Stoy (2008), T. Shimizu et
al. (2015)). However, sap flow measurements may be subject to various
potential sources of error. Some of these errors are related to scaling
sap flow variability both within trees and from tree to stand level (T.
J. Hatton, Moore, \& Reece (1995); Hernandez-Santana,
Hernandez-Hernandez, Vadeboncoeur, \& Asbjornsen (2015); Mitchell,
Irwin, Flanagan, \& Karron (2009)), while others are related to
intrinsic limitations of the methods or to how these methods are applied
(see Vandegehuchte \& Steppe (2013)). Although these biases have been
studied, they have not yet been quantified globally and there is no
conclusive assessment of how they differ across methods or species
characteristics, including wood properties (Poyatos et al. (2016)).\par

Sap flow methods (Vandegehuchte \& Steppe (2013)) can measure sap flow
rate (SF, g h-1 or equivalent units) or sap flux density (i.e., sap flow
rate per unit sapwood area, SFD, cm3 cm-2 h-1 or equivalent units) in a
plant's conductive tissue and can be classified in four families
depending on how they heat the sapwood and how they measure sapwood
temperatures: (1) the Dissipation family, including thermal dissipation
(TD; Granier \& Une (1985)) and transient thermal dissipation (Do \&
Rocheteau (2002), TTD ({\textbf{???}})) methods, which measure the
dissipation of heat from a heated probe inserted in the sapwood with
reference to a reference, non-heated probe; (2) the Pulse family,
including the compensation heat pulse (CHP; Swanson \& Whitfield
(1981)), heat ratio (HR; S. S. Burgess et al. (2001)), T-max (Y. Cohen,
Fuchs, \& Green (1981)), calibrated average gradient (CAG; Testi \&
Villalobos (2009)), sapflow+ (SF+; Vandegehuchte \& Steppe (2012b),
Vandegehuchte, Steppe, \& Phillips (2012)), single probe heat pulse
(SPHP; ({\textbf{???}})) and dual heat pulse methods (Dual; Pearsall,
Williams, Castorani, Bleby, \& Mcelrone (2014)), which all apply heat in
pulses and track sapwood temperature changes caused by thermal
convection and conduction; (3) the Field family, including the heat
field deformation (HFD; N. Nadezhdina (2018); N. Nadezhdina, Cermák, \&
Nadezhdin (1998)) and its derivatives, which measure the shape changes
of a continuous heat field in the sapwood, using axial and tangential
probes; and (4) the Balance family, represented by stem heat balance
(SHB; ({\textbf{???}}); Sakuratani (1981); ({\textbf{???}})) and trunk
heat balance (THB; Čermák, Kučera, \& Nadezhdina (2004); Čermák1973)
methods, which measure the energy balance across a heated wood section.
This latter family is the only one directly measuring sap flow rate,
while all the others measure sap flux density.\par

Methodological errors in sap flux density measurements may be caused by
wounding following probe insertion into the sapwood (except for the
miniaturized non-invasive ones; see (Michael J. Clearwater, Luo, Mazzeo,
\& Dichio (2009); Hanssens, De Swaef, Nadezhdina, \& Steppe (2013);
Schreel \& Steppe (2018)), biological variation in wood parameters and
diverse raw data processing approaches (A. C. Oishi, Hawthorne, \& Oren
(2016); Peters et al. (2018); Vergeynst, Vandegehuchte, McGuire, Teskey,
\& Steppe (2014)). Wounding affects heat and water transport and thus
may disrupt sap flow measurements (Barrett, Hatton, Ash, \& Ball (1995);
S. S. Burgess et al. (2001); S. Green, Clothier, \& Perie (2009); Steve
Green, Clothier, \& Jardine (2003); S. R. Green \& Clothier (1988);
Steppe, Vandegehuchte, Tognetti, \& Mencuccini (2015)), especially
during long-term installations (S. Marañón-Jiménez et al. (2018); A
Wiedemann, Jiménez, Rebman, Cuntz, \& Herbst (2013)). While wound
corrections have been available for a long time for some Pulse family
methods (Steve Green et al. (2003); Swanson \& Whitfield (1981)), they
have only become recently available for other methods such as TD
(Andreas Wiedemann et al. (2016)). Sap flux density methods are also
affected by changes in radial patterns, which are not constant over
time, so these methods have to measure the entire sapwood depth by
sufficiently large probes, or by individual measurement points at
different depths (T. Hatton (1990)). Although some methods have a more
solid theoretical background based on the physics of thermal transport
(Pulse and Balance methods), all of them rely on a certain degree of
empiricism, which may introduce errors caused by biological variability
and/or variation in signal processing approaches: species-specific
empirical calibrations in Dissipation methods (S. Fuchs, Leuschner,
Link, Coners, \& Schuldt (2017)), zero-flow determination or baselining
(Ping Lu, Urban, \& Zhao (2004); Peters et al. (2018)), and different
parameterization of thermal sapwood properties in Pulse and Field
methods (Chen, Miller, Rubin, \& Baldocchi (2012)). These thermal
sapwood parameters could change over time, introducing further errors in
the measurements (e.g.~changes in stem water content; Vergeynst et al.
(2014)). Within the Balance family, those using external heating do not
suffer from potential errors due to wounding, but they all require
zero-flow determination (D. Smith \& Allen (1996)). Balance methods have
often been considered to better integrate spatial variability in sap
flow (Čermák et al. (2004)), but whether they perform generally better
than sap flux density methods remains unknown.\par

Other errors in sap flow measurement may result from not accounting
properly for method-specific assumptions. For many sap flow methods,
natural temperature gradients (NTG) need to be minimized and/or
accounted for to obtain unbiased estimates of sap flow (Reyes-Acosta,
Vandegehuchte, Steppe, \& Lubczynski (2012); Vandegehuchte, Burgess,
Downey, \& Steppe (2015)). Incorrect sensor geometry (misalignment)
affects the accuracy of the measurements (S. S. Burgess et al. (2001);
Cabibel \& F (1991); Ren et al. (2017); Swanson (1983); Swanson \&
Whitfield (1981)). Other application errors, such as those arising from
the incomplete contact of TD probes with the sapwood (Michael J
Clearwater, Meinzer, Andrade, Goldstein, \& Holbrook (1999)) are
difficult to prevent, though they can be reasonably corrected a
posteriori (e.g.~Clearwater correction, K. R. Hultine et al. (2010);
Paudel, Kanety, \& Cohen (2013)). Despite that these application errors
have been well described in individual studies, a general quantification
of these errors for the most employed sap flow methods is currently
lacking.\par

Comparisons of sap flow measurements with respect to a reference method
(hereafter, for simplicity, `sap flow calibrations') are usually aimed
at obtaining species-specific calibrations (Vandegehuchte \& Steppe
(2013)) to assess different parameterizations of wood thermal properties
(Vandegehuchte \& Steppe (2012a)) or to validate empirical corrections
(e.g., wounding, NTG, changes in water content, misalignment; S. S.
Burgess et al. (2001); Vergeynst et al. (2014)). Although few studies
calibrate multiple sap flow methods for different species (S. Fuchs et
al. (2017)), collectively these calibration studies have shown the
inherent limitations of different sap flow methods to deal with low
(Steve Green et al. (2003)) or high flows (S. Green et al. (2009)).
Variability and quality in calibration performance may also be related
to specific wood properties such as wood density (Suleiman, Larfeldt,
Leckner, \& Gustavssor (1999); Wullschleger, Childs, King, \& Hanson
(2011)), especially given the fact that wood density enters the
calculation of sap flux density for some methods (Vandegehuchte \&
Steppe (2012a)) and co-varies with wood moisture content (Looker,
Martin, Jencso, \& Hu (2016)). Because thermal properties of wood are
dependent on both, i.e.~wood density and moisture content (MacLean
(1941)), they might additionally be influenced by wood anatomical traits
such as wood porosity type, i.e.~coniferous, diffuse-porous or
ring-porous wood. In conifers, however, no clear effects of wood density
on calibration variability have been reported (Peters et al. (2018)).
Therefore, a quantitative synthesis of sap flow calibrations, accounting
for variation caused by different flow ranges and wood properties is
needed to generalize and understand the patterns observed in individual
calibration studies.\par

Here, we compile a global database of published sap flow calibrations to
quantify the measurement errors associated with different sap flow
methods and to assess the factors underlying variability across methods.
In assessing calibrations, we distinguished between mean systematic bias
(accuracy), a measure of the average degree of closeness of the
measurements to the value obtained with a reference method; proportional
bias, which occurs when the magnitude of the error is a function of the
flow; linearity in the relationship between measurements and values
obtained with a reference method; and precision, a measure of
reproducibility and repeatability. Our main objective is to assess the
differences in accuracy, proportional bias, linearity and precision
among methodological families and individual sap flow methods; in
addition, we will determine whether calibration performance across
methods is associated with species wood traits (wood density and
porosity type).\par

\section{Material and methods}\label{material-and-methods}

\subsection{Sap flow calibration
datasets}\label{sap-flow-calibration-datasets}

We retrieved sap flow calibration studies of the seven most common
methods (CHP, T-max, HR, HFD, SHB, TD, TTD) applied on trees, palms or
lianas, using standard database searching tools (i.e.~Scopus, Web of
Science and Google Scholar). The search was conducted in June 2017
applying the following keywords: sap fl*, sap flux density, calibration,
potomet*, gravimet*, thermal dissipation, heat pulse, heat balance, heat
field deformation, compensation heat pulse, T-max, and their
combinations. Other sources of data were obtained from the references of
previously collected studies. For each calibration experiment, we
obtained paired observations of sap flow, measured with a thermal method
and with an independent reference method (typically gravimetric or
volumetric). Data was digitized from published figures (using GetData
Graph Digitalizer version 2.26.0.20). We asked the authors to supply the
raw data when these were unavailable from the original publication. We
obtained data from 60 studies (see Table A1) reporting 374 individual
calibration experiments performed on 81 different shrub and trees
species (10,186 data points in total). In the analysis, we only used
calibrations that were properly applied according to our definition
below (i.e.~290 calibrations out of 374) to restrict the variability to
the intrinsic characteristics of the methods.\par

We always considered sap flow observations obtained with the original
parameters of the methods (e.g.~Granier's original calibration for TD),
without applying the coefficients derived from the calibrations
themselves. Some calibrations with TD gave measured K values (i.e.~sap
flow index, calculated from raw sapwood temperature differences) instead
of measured SFD and, in these cases, K values were transformed to SFD
using Granier's original equation and calibration coefficients (Eq. 1, a
= 42.84 \(cm^3 cm^{-2} h^{-1}\), b = 1.231) (Granier \& Une (1985)).
\begin{equation}
SFD = a\times K^b
\end{equation}
For each calibration, we recorded the type of calibration material:
whole plants, whole plants without roots or cut stem segments. We also
assessed whether the sap flow method was properly applied using the best
available protocol specified for each method. We considered a proper
application of Dissipation methods when the probe was shorter than the
sapwood depth and radial profile correction was applied, when the probe
was approximately equal to the sapwood depth, or when the probe was
longer than the sapwood depth and this effect was corrected for
following Clearwater et al. (1999). To test whether our results could
have been affected by this correction, we performed a preliminary
analysis with the same structure as the main statistical model
(cf.~section 1.3.3) comparing TD calibrations with or without the
Clearwater correction and we did not find significant effects on any of
the metrics of calibration performance (cf.~section 1.3.2) (data not
shown). We considered a proper application of Pulse methods when wound
correction was applied and either the probe had multiple measuring
points along the sapwood or a radial sap flow profile correction was
applied. We always considered Balance methods and Field methods as
properly applied, because they always integrate (or account for) spatial
variability of sap flow.\par

Finally, to analyze the influence of wood traits on calibration
performance we used wood density, defined as fresh volume over oven-dry
mass, and wood porosity type of the species employed in each study. Wood
density was supplied in only a few studies (7 species \textasciitilde{}
80 calibration experiments \textasciitilde{} 4 studies). Assuming that
for wood density between-species variability is typically larger than
within-species variability (Siefert et al., 2015; Vilà-Cabrera et al.,
2015), we retrieved wood density of each species from the TRY database
(Kattge et al., 2011). Wood densities of Carica papaya, Phoenix
dactylifera and Vitis vinifera were obtained from Kempe (2014), Fathi
(2014) and Castelan-Estrada (2002), respectively, as they were not
recorded in TRY. When wood density could not be found for a given
species, we used the phylogenetically nearest species of the same genus
if available in TRY (10 of 81 species; e.g.~Citrus sinensis for Citrus
reticulata). We could not estimate wood density for three taxa (Humulus
lupulus, Musa sp. and Siagrus romanzoffiana). A correlation between
calibration-specific and species-level wood density extracted from the
TRY database (r = 0.78, P \textless{} 0.01, n = 12 calibrations, 6
species) indicates that species-level wood density values indeed are
applicable for our purpose, but the results should be interpreted with
caution. Finally, wood porosity was obtained from the InsideWood
database (Wheeler (2011), ({\textbf{???}})), using four categories:
ring-porous, diffuse-porous (i.e.~diffuse and semi-diffuse porous),
conifer and monocots.\par

\subsection{Calibration assessment}\label{calibration-assessment}

Although the reference methods always provide an estimate of sap flow
through plants or stem segments, sap flow measurements can be reported
as SF or SFD. It was not possible for us to interconvert between SF and
SFD in all cases because sapwood areas were not always reported. This
precluded a joint analysis of all the paired observations in the same
linear model because units differ between SF and SFD. To overcome this
problem and to maximize the amount of data considered in the analyses,
we first evaluated calibration performance using four complementary
dimensionless metrics at the calibration level, which allowed us to
analyze globally all calibrations regardless of the magnitude they
reported (SF or SFD). In a second stage, we quantified the variability
in the absolute errors in sap flow measurements across methods and flow
ranges separately for SF and SFD methods. We did not expect differences
between calibrations reported with SF or SFD because the
inter-conversion between them only involves a scalar transformation. In
addition, preliminary analyses confirmed that there was no significant
difference between SF and SFD for any of the calibration performance
metrics reported in this study (Table A2).\par

For the global analysis, the following metrics were calculated for each
calibration (SF and SFD): the average ln ratio (Ln-Ratio) between
measured and reference values as a measure of overall accuracy; the
slope of the relationship between measured and reference sap flow to
characterize proportional bias (Slope); the slope of the ln-ln
relationship between measured and reference sap flow as a measure of
linearity (Slope (ln-ln)); and Z Pearson's Correlation to describe
precision (Z-Cor) (Fig. 1.1). To calculate these metrics, we filtered
out data points with measured or reference flows less or equal to 0. All
calibration metrics and subsequent statistical models were performed in
R 3.4.2 (R Core Team (2017)). For model-based metrics, we always checked
residuals to ensure they satisfied normality and homoscedasticity
assumptions. Accuracy was evaluated as the mean of the natural logarithm
of the ratio between paired measurements (j) of each calibration (i):
\begin{equation}
Ln-Ratio_i = \frac{\sum_{j=1}^{n} ln(\frac{measured_j}{reference_j})_i}{n_i}
\end{equation}
where \(measured_j\) and \(reference_j\) are the paired measurements of
sensor-estimated and reference flow, respectively, and n the number of
paired measurements for each calibration i (see Fig. 1.1). The Ln-Ratio
varies between \(-\infty\) and \(+\infty\), and equals 0 for a
calibration with perfect mean accuracy (i.e.~lack of systematic bias).
We also expressed this metric as the exponential of Ln-ratio minus one
multiplied by 100, as an indicator of accuracy deviation (in \%).\par

The slope of the linear relationship (Eq. 3) describes how the magnitude
of the error changes (linearly) as a function of the reference flow. The
slope of the ln-ln relationship (Eq. 4) captures the linearity between
the measured and the reference flow. Both slope estimates were
calculated for each calibration using a simple linear regression (lm -
package stats):
\begin{equation}
measured_{ij} \sim \beta_{0i}+\beta_{1i}\;reference_{ij}+e_{ij}
\end{equation}
\begin{equation}
ln(measured_{ij}) \sim \beta^{\prime}_{0i}+\beta^{\prime}_{1i}\;ln(reference_{ij}) +e_{ij}
\end{equation}
where \(\beta_{0i}\) and \(\beta'_{0i}\) are the intercepts and
\(\beta_{1i}\) and \(\beta'_{1i}\) are the slopes for each calibration
(\(i\)), and \(j\) indicates individual calibration points. Hereafter,
we will refer to \(\beta_{1i}\) as \(Slope\) and to \(\beta'_{1i}\) as
\(Slope (ln-ln)\); slope values equal to 1 characterize measurements
without proportional bias (Eq. 3) and with high linearity (Eq. 4),
respectively.\par

We used Pearson's correlation coefficients r between measured and
reference flow of each calibration experiment (\(i\)) as a metric to
describe the precision of the methods. The distribution of the resulting
variable was skewed due to the large amount of correlation coefficients
close to 1, so we used Fisher's Z transformation (Eq. 5) to achieve
normality:
\begin{equation}
Z-Cor_i = \frac{1}{2}ln(\frac{1+r_i}{1-r_i})
\end{equation}
Low values of \(Z-Cor\) correspond to low r correlations, and high
values of \(Z-Cor\) correspond to high \(r\) correlations and thus high
precision (data set range \(r\) = {[}0.0491 -- 0.9999{]}; \(r\)= 0.0491
\textasciitilde{} \(z\) = 0.0491; \(r\) = 0.9999 \textasciitilde{} \(z\)
= 5.1594).\par
\begin{figure}[hbt!]

{\centering \includegraphics[width=0.55\linewidth]{figure/CH2/PLOTMETRICS} 

}

\caption{Graphical representation of the calibration performance metrics used in the analyses. Each panel presents the same simulated calibration points, representing plausible data. Blue dots represent an accurate, unbiased, linear and precise calibration, while red dots represent an inaccurate, biased, non-linear and imprecise calibration.}\label{fig:ch1fig1}
\end{figure}
In the analysis of the absolute errors of sap flow measurements, we
calculated the Normalized Root Mean Square Error (NRMSE) for each
calibration (\(i\)) (Eq. 6), separately for SFD and SF methods, in order
to obtain the percentage of absolute error at the mean range of each
calibration (\(i\)).
\begin{equation}
NRMSE_i = \frac{(\sqrt \frac{\sum_{j=1}^{n} (measured_j-reference_j)^2}{n})_i\times 100}{range\;mean_i}
\end{equation}
Subsequently, from the NRMSE and the mean range of each calibration, we
fitted a linear model for each method allowing to quantify the absolute
error (RMSE) at a given sap flow and also to obtain a RMSE at a
reference flow (cf.~section 2.3).\par

\subsection{Statistical analyses}\label{statistical-analyses}

All the analyses were performed using linear mixed-effects models (LMM)
with the package lmer (Bates, Mächler, Bolker, \& Walker (2015)).
Least-square means were estimated with package lsmeans (Lenth (2016))
and used to summarise the effects of fixed factors and to test contrasts
among predictions. In all models, we used the variables Study and
Species as partially crossed random effects (Schielzeth \& Nakagawa
(2013)), as we are interested in taking into account the variability
associated with study and species, and also to analyze within- and
between-group variability. We used Study as we expect experimental
variability between researchers or laboratories, and Species because
calibration performance has been reported to vary across species (S.
Fuchs et al. (2017); D. Smith \& Allen (1996); Steppe et al. (2015)).
For each model, \(R^2_m\) and \(R^2_c\) (marginal and conditional
coefficients of determination, respectively) based on Nakagawa \&
Schielzeth (2013) were calculated using the function r.squaredGLMM of
the package MuMIn (Barton (2017)) in R. Intraclass Correlation
Coefficients (ICC) were also calculated for the random factors to
quantify the proportion of variance within and among groups (low ICC
implies high intra-group variability).\par

In a first analysis, we were interested in assessing the differences in
calibration metrics (Ln-Ratio, Slope, Slope (ln-ln), Z-Cor) between
different families of methods (Family: Pulse, Dissipation, Balance and
Field methods), because methods within a family share similar physical
principles. We also analyzed differences between individual methods with
a sufficient sample size (Method: CHP, T-max, HR, HFD, SHB, TD, TTD). As
the calibration material determines, to a large extent, the experimental
conditions, we also included this variable in our models (Material:
whole plants, whole plants without roots or cut stem segments). For the
analysis of absolute errors of sap flow measurements, we modelled NRMSE
as a function of Method and the Mean Range of SFD (or SF for Balance
methods) in each calibration, as well as their interaction. We used the
same random structure as in previous models.\par 

Finally, we assessed how each calibration metric depended on Wood
Density and Wood Porosity. A first model included all methods available,
with Method interacting with Wood Density as predictors. In order to
test Wood Porosity effects, we fitted separate models for CHP and TD
calibrations, as these two methods were the only ones that had enough
data (\textgreater{} 5 calibrations) for more than one type of porosity.
Separate models were needed because not all wood porosity types were
represented for all methods. In both models we also included Material as
an explanatory cofactor, and the same random structure as in the first
analysis explained above.\par

\section{Results}\label{results}

Most of the published calibrations were performed with Pulse and
Dissipation methods (Table 1.1). In particular, 61\% of the total number
of the properly applied calibrations were conducted using TD or CHP,
followed by HFD and HR. SHB, T-max and TTD methods were less
represented, with 8 -- 14 calibrations each. The metrics extracted from
the raw calibrations were highly variable within methods (Fig. 1.2).
Calibration metrics often followed a quasi-normal distribution, but in
most cases distributions were truncated or skewed, particularly for
methods with fewer calibrations (Fig. 1.2).\par
\begin{table}[!h]

\caption{\label{tab:Ch1T1}Analysis summary for the different methods and families of methods obtained from the LMM models (least-squares means). We provide the four dimensionless metrics: the Ln-ratio as a measure of accuracy, the accuracy deviation calculated as the exponential of the Ln-ratio minus one multiplied by 100, the Slope to characterize the proportional bias, the slope of the ln-ln-relationship, Slope (ln-ln), as a measure of linearity, and Z Pearson’s correlation (Z-Cor) to describe overall precision; n: number of calibrations; studies: number of studies of each method; species: number of different species; r is the correlation calculated as the tanh of Z-Cor.}
\fontsize{6}{8}\selectfont
\begin{tabular}[t]{lllllllllll}
\toprule
Method & Family & n & studies & species & Ln-Ratio & Accuracy deviation \% & Slope & Slope (ln-ln) & Z-Cor & r\\
\midrule
CHP & Pulse & 63 & 16 & 21 & 0.133 & 14.225 & 0.887 & 0.783 & 1.837 & 0.950\\
T-max & Pulse & 11 & 5 & 6 & -0.053 & -5.162 & 0.614 & 0.697 & 1.755 & 0.942\\
HR & Pulse & 23 & 6 & 7 & -0.145 & -13.498 & 0.845 & 0.841 & 2.000 & 0.964\\
HFD & Field & 57 & 3 & 4 & -0.073 & -7.040 & 0.901 & 0.782 & 2.378 & 0.983\\
SHB & Balance & 8 & 5 & 6 & -0.242 & -21.494 & 0.847 & 0.967 & 2.287 & 0.980\\
TD & Dissipation & 115 & 18 & 35 & -0.519 & -40.488 & 0.683 & 1.066 & 1.711 & 0.937\\
TTD & Dissipation & 14 & 2 & 6 & -0.493 & -38.921 & 0.669 & 0.985 & 1.464 & 0.899\\
all & Pulse & 97 & NA & 30 & 0.012 & 1.167 & 0.844 & 0.787 & 1.874 & 0.954\\
all & Field & 57 & NA & 4 & -0.008 & -0.820 & 0.896 & 0.762 & 2.322 & 0.981\\
all & Balance & 8 & NA & 6 & -0.244 & -21.650 & 0.854 & 0.972 & 2.294 & 0.980\\
all & Dissipation & 129 & NA & 37 & -0.464 & -37.153 & 0.681 & 1.052 & 1.666 & 0.931\\
\bottomrule
\end{tabular}
\end{table}
\begin{figure}[hbt!]

{\centering \includegraphics[width=0.55\linewidth]{figure/CH2/Distribution-metrics-orderedV4} 

}

\caption{Distribution of the calibration performance metrics for each method. Dots represents the value of each individual calibration metric. Crosses represent the average of the metric for each method. Horizontal, dashed lines specify reference, perfect calibration values for a given metric.}\label{fig:ch1fig2}
\end{figure}
\subsection{Calibration performance compared among methods and families
of
methods}\label{calibration-performance-compared-among-methods-and-families-of-methods}

The average accuracy deviation across sap flow methods (properly
applied) ranged between 14.2\% for CHP and -40.5\% for TD (Table 1.1).
There were significant differences in accuracy (Ln-Ratio) among families
of methods and for methods but not for calibration materials (Fig. 1.3).
The Dissipation family in general and the TD and TTD methods in
particular were the only cases for which the Ln-Ratio was significantly
different from 0 (p\textless{}0.001) (Fig. 1.3), indicating systematic
bias (underestimation).\par

Proportional bias, estimated by Slope, varied among methods and families
of methods (p\textless{}0.01). Among families, Dissipation methods
showed a significantly smaller Slope than Pulse and Field methods (Fig.
1.3(a)), which was largely driven by the low value of TD (Fig. 1.3(b)).
Also, both Pulse and Dissipation families had slopes significantly
different from 1 (p\textless{}0.01 and p\textless{}0.001, respectively),
but only the slope of the TD method was significantly lower than 1
(p\textless{}0.001) (Fig. 1.3). As for the effects of calibration
material, calibrations made with whole plants had a significant
proportional bias (Slope \textless{} 1, p\textless{}0.001) (Fig. 1.3).
\begin{figure}[p]

{\centering \includegraphics[width=1\linewidth]{figure/CH2/FAMILYMETHODS} 

}

\caption{Predictions of the LMM models calculated from least-squares means of the four calibration metrics (Ln-Ratio as a proxy for mean accuracy, Slope for proportional bias, Slope (ln-ln) for linearity and Z-Cor for precision) for (a) different families of sap flow methods or for (b) different sap flow methods and for different calibration materials (Segment: stem segment; Whole plant: whole plant on a container or lysimeter; No-roots: whole plant without roots). 95\% confidence intervals of the estimates are also shown. Different letters indicate significant differences between factors levels evaluated with Tukey's test. Horizontal, dotted lines indicate reference, perfect calibration values for a given metric. Asterisks (*) indicate significant departure from those reference values.}\label{fig:ch1fig3}
\end{figure}
Calibration linearity, as denoted by Slope (ln-ln), varied across
methods and families of methods (p\textless{}0.001). We observed higher
values of Slope (ln-ln) for the TD method compared to CHP, T-max, HR and
HFD. Consistently, the Dissipation family in general also had a higher
Slope (ln-ln) than the Pulse and Field families (Fig. 1.3). CHP, T-max
and HFD (and Pulse and Field methods in general) had a Slope (ln-ln)
significantly lower than 1 (Table 1.1 and Fig. 1.3(b)), indicating a
convex relationship between reference and measured flow. Calibrations
performed with whole plants suffered from lack of linearity, indicated
by Slope (ln-ln) significantly lower than 1 (Fig. 1.3).\par

Precision (Z-Cor) was explained by both method and calibration material.
The HFD method (and Field methods in general) provided significantly
higher precision than either Pulse or Dissipation methods (particularly
CHP, TD and TTD) (Table 1.1 and Fig. 1.3(b)). Calibrations performed on
stem segments provided higher precision than those conducted on whole
plants (with or without roots) (Fig. 1.3).\par

In all the previous models, little variability was explained by species
(\(\tau 00\), species), relative to the higher variability associated to
Study, particularly for the Ln-Ratio and Z-Cor models (\(\tau 00\),
study, Table A3). This is consistent with the low ICC values observed
for the species factor, indicating that there is more variability within
than among species (Table A3).\par

In addition, the analysis of the normalized absolute error for the
different methods showed that NRMSE decreased linearly with increasing
measured sap flow in CHP, HFD and TTD methods and increased for T-max,
SHB and TD (Table 1.2 and Fig. A2). For HR the increase in NRMSE with
measured sap flow was not significant. For all the methods that measure
SFD, the absolute error at a typical flow of 25 cm3 cm-2 h-1 ranged
between 6.3 cm3 cm-2 h-1 for CHP and 10.7 cm3 cm-2 h-1 for the TTD
method (Table 1.2).\par
\begin{table}[!h]

\caption{\label{tab:Ch1T2}Error analysis of different sap flow methods. The normalized root mean square error (NRMSE) is modelled as a function of method and the mean flow range for each calibration (and their interaction) using a LMM model with the same random structure as the main models (cf. section 2.3). $\beta_0$ and $\beta_1$ are the corresponding intercepts and slopes, respectively ($\beta_0$ expressed as \% NRMSE; $\beta_1$ expressed as \% NRMSE per change in $cm^3$ $cm^{-2}$ $h^{-1}$ for SFD or as \% NRMSE per change in $cm^3$ $h^{-1}$ for SF). This linear model, was also used to calculate a reference NRMSE at a sap flux equivalent to the percentile 50 of the range of the data in the calibrations (SFD: 25 $cm^3$ $cm^{-2}$ $h^{-1}$; SF: 1300 $cm^3$ $h^{-1}$). The expected NRMSE and RMSE (in brackets, in $cm^3$ $cm^{-2}$ $h^{-1}$, except for SHB that is in $cm^3$ $h^{-1}$) at a typical flow are also given.}
\centering
\fontsize{10}{12}\selectfont
\begin{tabular}[t]{cccc}
\toprule
\multicolumn{1}{c}{ } & \multicolumn{3}{c}{NRMSE} \\
\cmidrule(l{3pt}r{3pt}){2-4}
Method & $\beta_0$\; \% & $\beta_1$ & reference NRMSE (and RMSE)\\
\midrule
CHP & 27.03*** & -0.08*** & 25.04\% (6.26)\\
T-max & 31.56. & 0.25*** & 37.83\% (9.46)\\
HR & 9.38 & 0.81 & 29.59\% (7.40)\\
HFD & 30.33*** & -0.12*** & 27.45\% (6.86)\\
SHB & 14.85 & 0.02*** & 42.95\% (558.36)\\
TD & 34.93*** & 0.10*** & 37.31\% (9.33)\\
TTD & 44.04*** & -0.04*** & 42.94\% (10.73)\\
\bottomrule
\multicolumn{4}{l}{\textsuperscript{} Statistical significant levels: "." p<0.1 ; "*" p<0.05; "**" p<0.01; "***"}\\
\multicolumn{4}{l}{p<0.001.}\\
\end{tabular}
\end{table}
\subsection{Influence of wood traits}\label{influence-of-wood-traits}

We did not find any significant influence of wood density on accuracy
and linearity metrics (Fig. 1.4). Nonetheless, we observed a negative
effect of wood density on proportional bias of HFD and TD calibrations
(p\textless{}0.05 and p\textless{}0.1, respectively). In addition, a
significant positive effect of wood density on the precision of HFD
measurements was observed (p\textless{}0.001), indicating that the
higher the wood density, the higher the precision (Fig. 1.4).\par

We did not find any significant difference among wood porosity types in
calibration metrics for studies using the TD or CHP methods.
Nevertheless, the non-linearity (Slope (ln-ln) \textless{} 1) observed
in general for the CHP method (Fig. 1.3(b)) was only significant for
species with diffuse-porous wood, not for conifer species (Table
1.3).\par
\begin{figure}[hbt!]

{\centering \includegraphics[width=0.55\linewidth]{figure/CH2/DENSITY} 

}

\caption{Relationship between the four calibration performance metrics (Ln-Ratio as a proxy for accuracy, Slope for proportional bias, Slope (ln-ln) for linearity, and Z-Cor for precision) and wood density, for different sap flow methods. Horizontal, dashed red lines indicate reference, perfect calibration values for a given metric. Regression lines are shown for significant effects only, and the corresponding level of significance (p-value: <0.1: (.), <0.05: (*), < 0.01: (**), < 0.001: (***)) is also reported }\label{fig:ch1fig4}
\end{figure}
\begin{table}[!h]

\caption{\label{tab:Ch1T3}Least-squares means and 95\% CI calculated from the LMM models testing the effect of different wood porosity types (Wood porosity) on sap flow calibration performance metrics (Ln-Ratio as a proxy for accuracy, Slope for proportional bias, Slope (ln-ln) for linearity, and Z-Cor for precision) for CHP and TD methods. No differences were detected among wood anatomies. Significance levels indicate departure from an ideal calibration (Ln-Ratio = 0; Slope = 1; Slope (ln-ln) = 1)}
\centering
\fontsize{7}{9}\selectfont
\begin{tabular}[t]{ccccccc}
\toprule
\multicolumn{1}{c}{ } & \multicolumn{1}{c}{ } & \multicolumn{1}{c}{ } & \multicolumn{1}{c}{Accuracy} & \multicolumn{1}{c}{Proportional bias} & \multicolumn{1}{c}{Linearity} & \multicolumn{1}{c}{Precision} \\
\cmidrule(l{3pt}r{3pt}){4-4} \cmidrule(l{3pt}r{3pt}){5-5} \cmidrule(l{3pt}r{3pt}){6-6} \cmidrule(l{3pt}r{3pt}){7-7}
Method & Wood porosity & n & Ln-Ratio & Slope & Slope (ln-ln) & Z-Cor\\
\midrule
CHP & diffuse & 48 & -0.055 [-0.370 , 0.259] & 0.795 [0.557 , 1.033]. & 0.799 [0.671, 0.927] \*\* & 1.980 [1.729 , 2.231]\\
CHP & conifers & 15 & 0.066 [-0.603 , 0.734] & 1.043 [0.459 , 1.627] & 0.777 [0.469 , 1.085] & 1.381 [0.775 , 1.988]\\
TD & diffuse & 81 & -0.273 [-0.658 , 0.111] & 0.752 [0.491 , 1.014]. & 1.126 [0.917 , 1.336] & 1.672 [1.085 , 2.260]\\
TD & ring & 16 & -0.405 [-0.866 , 0.056]. & 0.743 [0.410 , 1.077]. & 0.984 [0.681 , 1.286] & 2.260 [1.572 , 2.947]\\
TD & conifers & 15 & -0.396 [-0.873 , 0.080]. & 0.808 [0.468 , 1.147] & 1.142[0.842 , 1.441] & 1.606 [0.892 , 2.321]\\
\bottomrule
\multicolumn{7}{l}{\textsuperscript{} Statistical significant levels: "." p<0.1 ; "*" p<0.05; "**" p<0.01; "***" p<0.001.}\\
\end{tabular}
\end{table}
\section{Discussion}\label{discussion}

Our results show a large variability in the quality of sap flow
calibrations, even within the same sap flow method (Fig. 1.2),
highlighting the large variability among and even within studies. This
implies that, even if methods are properly applied (as defined in
section 2.1), sap flow measurements can still produce biased estimates
of water transport rates in plants, and these errors will need to be
considered in quantitative analyses based on this type of measurements.
On average, however, all sap flow methods assessed here produced results
that may be acceptable for qualitative use in most applications, as
shown by the typical high correlation between measured and reference
values (r \textgreater{} 0.89 for all methods and method families, Table
1.1). For quantitative use, no method appears to be suitable for all
experimental contexts, and researchers need to consider both the
inherent limitations of the methods and the need to perform
study-specific calibrations (see Implications and recommendations, Table
1.4).\par

\subsection{Sap flow measurement errors across methods and
methodological
families}\label{sap-flow-measurement-errors-across-methods-and-methodological-families}

A relatively small part of the total variability in the quality of
calibrations is related to methods and families of methods and, to a
lesser extent, to the calibration material (fixed effects explain 8 --
28\% of the variability in calibration metrics; see \(R^2_m\) values in
Table A3). Despite the high variability within methods, we detected
significant differences between methods. Dissipation methods were the
only methods for which accuracy was significantly lower than expected
for an ideal calibration. This is consistent with previous reports
(Braun \& Schmid (1999); S. E. Bush, Hultine, Sperry, Ehleringer, \&
Phillips (2010); Caterina, Will, Turton, Wilson, \& Zou (2014);
({\textbf{???}}); de Oliveira Reis, Campostrini, Sousa, \& Silva (2006);
S. Fuchs et al. (2017); Ping Lu \& Chacko (1998); McCulloh et al.
(2007); Montague \& Kjelgren (2006); Rubilar, Hubbard, Yañez, Medina, \&
Valenzuela (2017); Steppe, De Pauw, Doody, \& Teskey (2010); Taneda \&
Sperry (2008); Uddling, Teclaw, Pregitzer, \& Ellsworth (2009)) and our
synthesis confirms that most of the individual TD and all TTD
calibrations underestimate sap flow systematically (Fig. 1.5).
Interestingly, however, other studies have found the opposite result
(Cain (2009); K. R. Hultine et al. (2010); P Lu (2002); Sperling et al.
(2012); Sun, Aubrey, \& Teskey (2012)) and simulation models (Hölttä,
Linkosalo, Riikonen, Sevanto, \& Nikinmaa (2015); Wullschleger et al.
(2011)) suggest that it is difficult to state a priori whether TD will
over- or underestimate flow, as the measurements obtained are highly
dependent on wood properties and on flux conditions. Our results show
that, globally, the conditions leading to underestimation are more
frequent and support the existence of a proportional bias underlying
this systematic underestimation by TD (Fig. 1.3). It must be also noted,
however, that Dissipation methods have been tested against a much wider
range of flow conditions compared to the rest of the methods (Fig. 1.5,
Fig.6).\par
\begin{table}[!h]
\begin{threeparttable}
\caption{\label{tab:Ch1T4}Synthesis of the potential sources of error and use adequacy for each method. Crosses indicates that the method is sensitive to the respective source of error (updated from Vandegehuchte and Steppe, 2013). Methods are classified according to their use effectiveness under different flow conditions: dark grey, light grey and white indicate highly, partially and no recommended use, respectively. When assessing use adequacy for high/low flows, dark and light gray indicate a Normalized Root Mean Square Error (NRMSE) less than a 22\% and a 44\%, respectively, calculated with the NRMSE model (Table 1.2). In Absolute flows use recommendation, dark grey shows methods with both accuracy (Ln-Ratio) and proportional bias (Slope) not significantly different from a perfect calibration. In Relative flows use recommendation, dark grey shows methods with linearity (Slope (ln-ln)) not significantly different from a perfect calibration and with reasonable precision. Potentiality of measuring small stems diameters (< 125mm) is also reported.}
\centering
\fontsize{10}{12}\selectfont
\begin{tabular}[t]{cccccccccccccccc}
\toprule
\multicolumn{1}{c}{ } & \multicolumn{9}{c}{Potential source of measure error} & \multicolumn{6}{c}{Effectiveness in measuring} \\
\cmidrule(l{3pt}r{3pt}){2-10} \cmidrule(l{3pt}r{3pt}){11-16}
\rotatebox{90}{\em{Method}} & \rotatebox{90}{\em{Wounding}} & \rotatebox{90}{\em{Radial velocity profile}} & \rotatebox{90}{\em{Wood properties}} & \rotatebox{90}{\em{Natural thermal gradients}} & \rotatebox{90}{\em{Sensor installation}} & \rotatebox{90}{\em{Sensor design}} & \rotatebox{90}{\em{Baselining}} & \rotatebox{90}{\em{Power input}} & \rotatebox{90}{\em{Pulse length}} & \rotatebox{90}{\em{Reverse flows}} & \rotatebox{90}{\em{Low flows*}} & \rotatebox{90}{\em{High flows*}} & \rotatebox{90}{\em{Absolute flows}} & \rotatebox{90}{\em{Relative flows}} & \rotatebox{90}{\em{Small stems}}\\
\midrule
CHP & x & x & x & x & x &  &  &  & x & \multicolumn{1}{c}{\cellcolor[HTML]{FFFFFF}{\textcolor[HTML]{FFFFFF}{0}}} & \multicolumn{1}{c}{\cellcolor[HTML]{BBBBBB}{\textcolor[HTML]{BBBBBB}{1}}} & \multicolumn{1}{c}{\cellcolor[HTML]{999999}{\textcolor[HTML]{999999}{2}}} & \multicolumn{1}{c}{\cellcolor[HTML]{999999}{\textcolor[HTML]{999999}{2}}} & \multicolumn{1}{c}{\cellcolor[HTML]{FFFFFF}{\textcolor[HTML]{FFFFFF}{0}}} & \multicolumn{1}{c}{\cellcolor[HTML]{FFFFFF}{\textcolor[HTML]{FFFFFF}{0}}}\\
T-max & x & x & x &  & x &  & x &  &  & \multicolumn{1}{c}{\cellcolor[HTML]{FFFFFF}{\textcolor[HTML]{FFFFFF}{0}}} & \multicolumn{1}{c}{\cellcolor[HTML]{BBBBBB}{\textcolor[HTML]{BBBBBB}{1}}} & \multicolumn{1}{c}{\cellcolor[HTML]{FFFFFF}{\textcolor[HTML]{FFFFFF}{0}}} & \multicolumn{1}{c}{\cellcolor[HTML]{999999}{\textcolor[HTML]{999999}{2}}} & \multicolumn{1}{c}{\cellcolor[HTML]{FFFFFF}{\textcolor[HTML]{FFFFFF}{0}}} & \multicolumn{1}{c}{\cellcolor[HTML]{FFFFFF}{\textcolor[HTML]{FFFFFF}{0}}}\\
HR & x & x & x &  & x &  &  &  &  & \multicolumn{1}{c}{\cellcolor[HTML]{999999}{\textcolor[HTML]{999999}{2}}} & \multicolumn{1}{c}{\cellcolor[HTML]{999999}{\textcolor[HTML]{999999}{2}}} & \multicolumn{1}{c}{\cellcolor[HTML]{FFFFFF}{\textcolor[HTML]{FFFFFF}{0}}} & \multicolumn{1}{c}{\cellcolor[HTML]{999999}{\textcolor[HTML]{999999}{2}}} & \multicolumn{1}{c}{\cellcolor[HTML]{999999}{\textcolor[HTML]{999999}{2}}} & \multicolumn{1}{c}{\cellcolor[HTML]{999999}{\textcolor[HTML]{999999}{2}}}\\
HFD & x & x &  & x & x & x & x & x &  & \multicolumn{1}{c}{\cellcolor[HTML]{999999}{\textcolor[HTML]{999999}{2}}} & \multicolumn{1}{c}{\cellcolor[HTML]{BBBBBB}{\textcolor[HTML]{BBBBBB}{1}}} & \multicolumn{1}{c}{\cellcolor[HTML]{999999}{\textcolor[HTML]{999999}{2}}} & \multicolumn{1}{c}{\cellcolor[HTML]{999999}{\textcolor[HTML]{999999}{2}}} & \multicolumn{1}{c}{\cellcolor[HTML]{FFFFFF}{\textcolor[HTML]{FFFFFF}{0}}} & \multicolumn{1}{c}{\cellcolor[HTML]{999999}{\textcolor[HTML]{999999}{2}}}\\
SHB &  &  &  & x &  &  & x & x &  & \multicolumn{1}{c}{\cellcolor[HTML]{999999}{\textcolor[HTML]{999999}{2}}} & \multicolumn{1}{c}{\cellcolor[HTML]{999999}{\textcolor[HTML]{999999}{2}}} & \multicolumn{1}{c}{\cellcolor[HTML]{FFFFFF}{\textcolor[HTML]{FFFFFF}{0}}} & \multicolumn{1}{c}{\cellcolor[HTML]{999999}{\textcolor[HTML]{999999}{2}}} & \multicolumn{1}{c}{\cellcolor[HTML]{999999}{\textcolor[HTML]{999999}{2}}} & \multicolumn{1}{c}{\cellcolor[HTML]{999999}{\textcolor[HTML]{999999}{2}}}\\
TD & x & x & x & x &  & x & x & x &  & \multicolumn{1}{c}{\cellcolor[HTML]{FFFFFF}{\textcolor[HTML]{FFFFFF}{0}}} & \multicolumn{1}{c}{\cellcolor[HTML]{BBBBBB}{\textcolor[HTML]{BBBBBB}{1}}} & \multicolumn{1}{c}{\cellcolor[HTML]{BBBBBB}{\textcolor[HTML]{BBBBBB}{1}}} & \multicolumn{1}{c}{\cellcolor[HTML]{FFFFFF}{\textcolor[HTML]{FFFFFF}{0}}} & \multicolumn{1}{c}{\cellcolor[HTML]{999999}{\textcolor[HTML]{999999}{2}}} & \multicolumn{1}{c}{\cellcolor[HTML]{FFFFFF}{\textcolor[HTML]{FFFFFF}{0}}}\\
TTD & x & x & x & x &  & x & x & x &  & \multicolumn{1}{c}{\cellcolor[HTML]{FFFFFF}{\textcolor[HTML]{FFFFFF}{0}}} & \multicolumn{1}{c}{\cellcolor[HTML]{FFFFFF}{\textcolor[HTML]{FFFFFF}{0}}} & \multicolumn{1}{c}{\cellcolor[HTML]{BBBBBB}{\textcolor[HTML]{BBBBBB}{1}}} & \multicolumn{1}{c}{\cellcolor[HTML]{FFFFFF}{\textcolor[HTML]{FFFFFF}{0}}} & \multicolumn{1}{c}{\cellcolor[HTML]{999999}{\textcolor[HTML]{999999}{2}}} & \multicolumn{1}{c}{\cellcolor[HTML]{FFFFFF}{\textcolor[HTML]{FFFFFF}{0}}}\\
\bottomrule
\end{tabular}
\begin{tablenotes}
\small
\item [*] (Low/High: SFD methods: <5 / >80 $cm^{3} cm^{-2} h^{-1}$; SF methods: <260 / >3900 $cm^3 h^{-1}$)
\end{tablenotes}
\end{threeparttable}
\end{table}
Calibration parameters of the TD method were originally considered to be
universal but subsequent studies have claimed that species-specific
calibrations are necessary to obtain correct sap flow measurements (S.
Fuchs et al. (2017); Ping Lu et al. (2004); Steppe et al. (2010)). For a
set of diffuse-porous species, using a pooled calibration also
substantially improved TD (but not HFD) performance compared to
measurements obtained with the original calibration (S. Fuchs et al.
(2017)). However, our results show that species in general and wood
porosity type in particular explain a small or even no proportion of the
variability in the calibrations (Table 1.3 and A3). This implies that
factors related to the experimental context and, possibly, to
intraspecific variability in wood properties (cf.~section 1.5.2) may
have a large contribution to overall uncertainty. Therefore, our results
suggest that calibration parameters for TD or HFD, obtained under
different experimental contexts, may not be generalizable to species
level, as also suggested by Fuchs et al. (2017).\par

In addition to Dissipation methods, Pulse methods also suffer
proportional bias, probably driven by overestimation at low flows,
although this was significant for T-max only (i.e.~positive intercepts
in linear models fitted to calibration data; Fig. 1.5 and 1.6 and A3).
It is well known that the equations of CHP and T-max cannot be solved at
sap flows close to 0, and the calibration intercepts observed here (Fig.
A3) are consistent with the detection thresholds reported for T-max
(\textasciitilde{}10 cm3 cm-2 h-1; Steve Green et al. (2003)) and CHP
(2-4 cm3 cm-2 h-1; T. M. Bleby, Burgess, \& Adams (2004); Steve Green et
al. (2003)). Our results confirm and generalize a previously reported
low-flow detectability problem for T-max (S. Green et al. (2009), Steve
Green et al. (2003); Vandegehuchte \& Steppe (2012b)), but we could not
confirm it for CHP as described before (Barrett et al. (1995); Becker
(1998); T. M. Bleby et al. (2004); Vandegehuchte \& Steppe (2012b)).
Despite overestimation at low flows, the average accuracy of CHP and
T-max is good, which implies that low-flow overestimations may be
compensated with underestimations at high flows. This is also shown by
the lack of linearity observed in both methods (Slope (ln-ln)
\textless{} 1; Fig. 1.3(b)).\par

Our analysis did not detect the saturation effect for the HR method at
high flows that has been reported elsewhere (T. Bleby, McElrone, \&
Burgess (2008); S. Green et al. (2009); Steppe et al. (2015)). This is
likely due to the fact that HR calibrations considered here include few
observations in the region where this overestimation occurs
(\textgreater{} \textasciitilde{}45 cm3 cm-2 h-1; Figs 5 and 6).
Moreover, the high variability in the calibrations probably precluded
detection of the saturation effect (Fig. 1.3 and 1.5) and of the
apparent trend of increasing NRMSE with sap flow range for HR (Table 1.3
and Fig. A2). A lack of linearity can also be observed for HFD,
consistent with the suggested tendency of this method to underestimate
at high flows (Vandegehuchte \& Steppe (2012c)).\par
\begin{figure}[p]

{\centering \includegraphics[width=1\linewidth]{figure/CH2/figure-density} 

}

\caption{Relationship between measured and reference sap-flux density (SFD) for different sap flow methods, studies and calibrations. The fits of ln-ln regressions (Eq. 4) for each calibration are also depicted. Different colors represent different studies that report results in sap-flux density units. Scales vary across panels to facilitate intra method comparison. The red dotted line indicates the 1:1 relationship.}\label{fig:ch1fig5}
\end{figure}
Despite the large variability in precision within methods, our results
show that calibrations performed with HFD give more precise results than
those conducted using the CHP, TD and TTD methods. Although this result
should be interpreted with care as it is based on 57 calibrations but
only from 3 studies, the higher precision observed with HFD could lie in
the second dimension included in the method, which could better capture
the effect of anisotropy of the wood structure. This would also be
consistent with the fact that SHB, a method that is assumed to integrate
sap flow variability within the stem, was the method with the second
highest precision on average, albeit precision was very variable for
this method (Fig. 1.3(b)).\par

We did not detect differences in accuracy, proportional bias or
linearity of the calibrations across calibration materials. However,
compared to an ideal calibration, we did find proportional bias and lack
of linearity in calibrations performed on whole plants, probably because
these calibrations use large scales whose sensitivity and resolution are
usually low, potentially affecting low-flow measurements and leading to
artefactual overestimation at low flows. Poor linearity may also be due
to non-linear changes in belowground hydraulic resistance as the sap
flow increases (J. Martínez-Vilalta, Korakaki, Vanderklein, \&
Mencuccini (2007)). In cut plants, we may have two opposite effects, as
cutting could eliminate belowground resistance (favoring flow) but add
resistance due to putative embolism formation after cutting. Similarly,
the higher precision of calibrations conducted on cut stems relative to
those conducted on whole plants (with and without roots), likely
reflects that cut stem calibrations are normally conducted in
laboratories with precision scales and under controlled conditions that
minimize experimental random errors.\par
\begin{figure}[p]

{\centering \includegraphics[width=1\linewidth]{figure/CH2/figure-totalflow} 

}

\caption{Relationship between measured and reference sap flow (SF) for different sap flow methods, studies and calibrations. The fits of ln-ln regressions (Eq. 4) for each calibration are also depicted. Different color symbols and line types represent different studies. Scales varies across panels to facilitate intra method comparison. Insets are shown in some panels (T-max, SHB, TD) to facilitate visualization when the flow ranges differed markedly among calibrations for the same method. The red dotted line indicates 1:1 relationship.}\label{fig:ch1fig6}
\end{figure}
\subsection{The performance of sap flow calibrations is largely
unrelated to species wood
traits}\label{the-performance-of-sap-flow-calibrations-is-largely-unrelated-to-species-wood-traits}

Species-specific wood density and wood porosity type explained little
variability in overall calibration performance, although we detected
some effects of wood density for HFD and TD calibrations. Wood density
affected HFD measurements by increasing precision, which could be
related to the response time of the sensors. If we assume that maximum
sapwood water content is reduced as wood density increases (Simpson
(1993)), associated changes in thermal diffusivity could lead to a
faster sensor response (Hölttä et al. (2015)), higher correlation
between actual and measured flows. Wood density also showed a negative
relationship with proportional bias for HFD and TD, a pattern that could
be caused by the combined effects of wood density and water content on
wood thermal diffusivity (Vandegehuchte \& Steppe (2012c); Vergeynst et
al. (2014)). The fact that we did not find clear effects of wood density
on calibration accuracy and linearity, despite that wood density affects
thermal diffusivity and hence heat transport (Wullschleger et al.
(2011)), could be explained by two reasons. Firstly, we could not use
the actual wood density for most of the calibrations, because it was not
reported in the corresponding studies, and using species-level averages
instead of the wood density of the plant material specifically used on
the calibrations may mask the effect of wood density on calibration
performance. Secondly, wood density in angiosperms appears to be only
weakly correlated to some wood properties that could be important for
sap flow calibrations, such as lumen fraction (Zanne et al. (2010)).\par

Our global analysis did not show clear and consistent differences in
calibration quality between different wood porosity types (Table 1.2) as
previously suggested by several studies for both CHP (S. R. Green \&
Clothier (1988)) and TD methods (S. E. Bush et al. (2010); Sun et al.
(2012)). According to heat transport theory, we should expect declining
performance from conifer to ring-porous species (i.e.~from most
homogenous to most heterogeneous wood). For CHP, we found that
proportional bias (only marginally) and nonlinearity departed from an
ideal calibration for diffuse-porous species, but these patterns did not
differ significantly from those observed for conifers. Our results did
not clearly support either an inferior performance of TD in ring-porous
species compared to diffuse-porous or conifers, as could be expected
from the reported underestimation driven by large sap flow gradients
along sensor length or by the imperfect probe contact with hydroactive
xylem in species with narrow sapwood (Michael J Clearwater et al.
(1999); but see Wullschleger et al. (2011)). Wood porosity effects on
sap flow calibrations have been inferred in individual studies from
measurements in few species representative of each wood porosity type
(S. E. Bush et al. (2010); Sun et al. (2012); Xie \& Wan (2018)) and our
inability to detect these effects here may be caused by the high
variability in experimental context within our dataset. Furthermore, the
effect of the different anatomies may be masked by high structural
variability within wood porosity types, as for example the variation in
latewood to earlywood in conifers (Fan, Guyot, Ostergaard, \& Lockington
(2018)) and we cannot discard that calibration performance could be
related to quantitative anatomical traits not assessed in this study
(cf. Xie \& Wan (2018)). Although the low variability we observed at the
species level suggests that quantitative anatomical traits might not
explain much of the variability in sap flow calibrations, we encourage
that quantitative wood traits are measured in the same plant material
used to calibrate sap flow sensors to better understand the influence of
wood properties on the variability of sap flow calibrations.\par

\subsection{Implications and
recommendations}\label{implications-and-recommendations}

Our global analysis shows that even when the methods are applied
following standard recommendations the quality of individual
calibrations can be very low (Fig. 1.3). This result reflects, on one
hand, systematic bias in TD and lack of linearity in CHP, two of the
most widely used methods (Fig. A1) and, on the other hand, unknown
sources of error related to experimental conditions and/or sample
characteristics (Table 1.4). In our study, we could not account for all
the experimental conditions to evaluate these sources of variability,
except for the effect of the calibration material. Examples of factors
that may affect calibrations when using the same type of calibration
material include sensor design (S. Fuchs et al. (2017)), sensor
installation (T. M. Bleby et al. (2004); Ping Lu \& Chacko (1998); Ren
et al. (2017)), variation in calculations of wood thermal properties
(Looker et al. (2016)), zero flow determination (Looker et al. (2016);
Peters et al. (2018)) or the mechanism of flow generation in cut stem
calibrations (negative vs positive pressures) (S. Fuchs et al. (2017))
(Table 1.4). Previous reports, however, usually focus on only one of the
sources of experimental error. Importantly, relevant methodological
information that could be used to assess (and account for) these sources
of error is frequently not reported (Peters et al. (2018); Steppe et al.
(2015)). Clearly, further research into the effects of experimental
conditions on the quality of different sap flow methods should be a
priority, as well as more complete, standardized reporting of
experimental conditions, including information on the sources of
potential methodological errors listed in Table 1.4.\par

Our results show that calibrations may be needed to obtain correct
absolute values of sap flow, even when Pulse methods are used (see also
S. Fuchs et al. (2017); Steppe et al. (2010)). However, sap flow
calibrations provide a snapshot of the performance of a given sap flow
method under relatively stable conditions, which may greatly differ from
those experienced by plants in the field. Moreover, our analysis could
not address the methodological variability related to more dynamic
effects such as errors caused by changes in sapwood water content
(Vergeynst et al. (2014)), long-term wounding or signal dampening
(Maranon-jimenez2018; Peters et al. (2018); A Wiedemann et al. (2013)).
In this sense, more studies should assess calibration applicability to
mid- or long-term measurements (e.g., Oliveras \& Llorens (2001)),
possibly combined with independent estimates of sapwood water content
(Vandegehuchte \& Steppe (2012b)) and whether calibrations obtained from
excised segments are valid for whole-plants.\par

Considering only their performance in calibration tests (i.e.~no other
logistic or technical issues, such as sensor, datalogging, or power
constraints, which will be study-specific) we can provide some general
recommendations on the use of sap flow methods (Table 1.4). The most
widely used method, TD, appears to be consistently inaccurate, shows
proportional bias and generally underestimates sap flow, by 40\% on
average (if used with its original calibration coefficients). However,
it presents good linearity, which implies that this method can be used
when sap flow responses to environmental variables and/or treatments are
the primary focus of the study (i.e., good estimates of absolute sap
flow values are not critical). In comparison, CHP, T-max and HFD all
present a certain nonlinearity which may affect the estimation of these
environmental responses. At least for CHP and T-max (specially for the
latter) this pattern seems to be driven by overestimation at low flows
and underestimation at high flows canceling out each other. This implies
that both Pulse methods could be suitable for studies interested in
absolute values of transpiration. For the HFD method, the nonlinearity
could be influencing the estimations of radial sap flow patterns, as
these measurements would need to correctly measure both high and low
flows simultaneously. We also confirm that the HR method may not be
suitable to measure high flows but it is probably the best method for
detailed physiological studies involving low flows.\par

\section{Conclusions}\label{conclusions}

In conclusion, our global assessment contributes towards a proper
incorporation of measurement errors in the interpretation of individual
case studies and in modelling studies aimed at upscaling sap flow data
(T. J. Hatton et al. (1995); Hernandez-Santana et al. (2015)). Perhaps
even more importantly, it paves the way towards improved intercomparison
of sap flow datasets obtained with different methods to assess regional
or global patterns in plant water use (e.g., the SAPFLUXNET initiative;
Poyatos et al. (2016)). Although providing explicit correction factors
for each method is beyond the scope of this paper, the typical accuracy
deviations provided in Table 1.1 can be used as a first order correction
when combining sap flow data from different methods (and no additional
information on study-specific uncertainty sources is available).\par

\chapter*{References}\label{references}
\addcontentsline{toc}{chapter}{References}

\setlength{\parindent}{-0.20in} \setlength{\leftskip}{0.20in}
\setlength{\parskip}{8pt}

Abeli, T. et al. 2014. Effects of marginality on plant population
performance. - J. Biogeogr. 41: 239--249.\par
Ackerly, D. D. 2003. Community Assembly , Niche Conservatism , and
Adaptive Evolution in Changing Environments. - Int. J. Plant Sci. 164:
164--184.\par
Ackerly, D. D. et al. 2010. The geography of climate change:
Implications for conservation biogeography. - Divers. Distrib. 16:
476--487.\par
Adler, P. B. et al. 2013. Trait-based tests of coexistence mechanisms. -
Ecol. Lett. 16: 1294--1306.\par
AEMET (Spanish Meteorological Agency) 2014. Avance climatológico mensual
mes de septiembre 2014 en la Región de Murcia.\par
AEMET (Spanish Meteorological Agency) and IP (Portuguese Meteorological
Insitute) 2011. Iberian climate atlas. Air temperature and precipitation
(1971-2000) (MR y M Ministerio de Medio Ambiente, Ed.).\par
Aitken, S. N. et al. 2008. Adaptation, migration or extirpation: climate
change outcomes for tree populations. - Evol. Appl. 1: 95--111.\par
Alexander, J. M. et al. 2016. When Climate Reshuffles Competitors: A
Call for Experimental Macroecology. - Trends Ecol. Evol. 31:
831--841.\par
Allen, C. D. and Breshears, D. D. 1998. Drought-induced shift of a
forest -- woodland ecotone: Rapid landscape response to climate
variation. - Ecology 95: 14839--14842.\par
Allen, C. D. et al. 2010. A global overview of drought and heat-induced
tree mortality reveals emerging climate change risks for forests. - For.
Ecol. Manage. 259: 660--684.\par
Allen, C. D. et al. 2015. On underestimation of global vulnerability to
tree mortality and forest die-off from hotter drought in the
Anthropocene. - Ecosphere 6: art129.\par
Anderegg, W. R. L. et al. 2012. The roles of hydraulic and carbon stress
in a widespread climate-induced forest die-off. - Proc. Natl. Acad. Sci.
109: 233--237.\par
Anderson, R. P. et al. 2002. Using niche-based GIS modeling to test
geographic predictions of competitive exclusion and competitive release
in South American pocket mice. - Oikos 98: 3--16.\par
Anne, P. 1945. Carbone organique (total) du sol et de l'humus. - Ann.
Agron. 15: 161--172.\par
Araújo, M. B. and Guisan, A. 2006. Five (or so) challenges for species
distribution modelling. - J. Biogeogr. 33: 1677--1688.\par
Araújo, M. B. and New, M. 2007. Ensemble forecasting of species
distributions. - Trends Ecol. Evol. 22: 42--47.\par
Araújo, M. B. et al. 2013. Heat freezes niche evolution (D Sax, Ed.). -
Ecol. Lett. 16: 1206--1219.\par
Araújo, M. B. et al. 2019. Standards for distribution models in
biodiversity assessments. - Sci. Adv. 5: eaat4858.\par
Ashraf, M. et al. 2011. Drought Tolerance: Roles of Organic Osmolytes,
Growth Regulators, and Mineral Nutrients. - Adv. Agron. 111:
249--296.\par
Austin, M. 1971. The role of regression analysis in plant ecology. -
Proc. Ecol. Soc. Aust. 6: 63--75.\par
Austin, M. P. et al. 1990. Measurement of the Realized Qualitative
Niche: Environmental Niches of Five Eucalyptus Species. - Ecol. Monogr.
60: 161--177.\par
Barbet-Massin, M. et al. 2012. Selecting pseudo-absences for species
distribution models: How, where and how many? - Methods Ecol. Evol. 3:
327--338.\par
Barry, J. P. et al. 1995. Climate-Related, Long-Term Faunal Changes in a
California Rocky Intertidal Community. Science. 267: 672--675.\par
Barve, N. et al. 2011. The crucial role of the accessible area in
ecological niche modeling and species distribution modeling. - Ecol.
Modell. 222: 1810--1819.\par
Belyea, L. R. and Lancaster, J. 1999. Assembly Rules within a Contingent
Ecology. - Oikos 86: 402.\par
Benito Garzón, M. et al. 2011. Intra-specific variability and plasticity
influence potential tree species distributions under climate change. -
Glob. Ecol. Biogeogr. 20: 766--778.\par
Bernard-Verdier, M. et al. 2012. Community assembly along a soil depth
gradient: Contrasting patterns of plant trait convergence and divergence
in a Mediterranean rangeland (H Cornelissen, Ed.). - J. Ecol. 100:
1422--1433.\par
Bertrand, R. et al. 2012. Disregarding the edaphic dimension in species
distribution models leads to the omission of crucial spatial information
under climate change: The case of Quercus pubescens in France. - Glob.
Chang. Biol. 18: 2648--2660.\par
Bertrand, R. et al. 2016. Ecological constraints increase the climatic
debt in forests. - Nat. Commun. 7: 12643.\par
Bigler, C. et al. 2006. Drought as an inciting mortality factor in scots
pine stands of the Valais, Switzerland. - Ecosystems 9: 330--343.\par
Blonder, B. et al. 2014. The n-dimensional hypervolume. - Glob. Ecol.
Biogeogr. 23: 595--609.\par
Blonder, B. et al. 2015. Linking environmental filtering and
disequilibrium to biogeography with a community climate framework. -
Ecology 96: 972--985.\par
Blonder, B. et al. 2017. Predictability in community dynamics. - Ecol.
Lett. 20: 293--306.\par
Bowler, D. and Böhning-Gaese, K. 2017. Improving the
community-temperature index as a climate change indicator. - PLoS One
12: 1--17.\par
Boyce, M. S. et al. 2002. Evaluating resource selection functions. -
Ecol. Modell. 157: 281--300.\par
Bramer, I. et al. 2018. Advances in Monitoring and Modelling Climate at
Ecologically Relevant Scales. - Adv. Ecol. Res. 58: 101--161.\par
Braun-Blanquet, J. and Bolòs, O. 1957. The plant communities of the
Central Ebro Basin and their dynamics. - An. la Estac. Exp. Aula Dei 5:
1--266.\par
Breiner, F. T. et al. 2017. Including environmental niche information to
improve IUCN Red List assessments (B Schröder, Ed.). - Divers. Distrib.
23: 484--495.\par
Broennimann, O. and Guisan, A. 2008. Predicting current and future
biological invasions: both native and invaded ranges matter. - Biol.
Lett. 4: 585--9.\par
Broennimann, O. et al. 2006. Do geographic distribution, niche property
and life form explain plants' vulnerability to global change? - Glob.
Chang. Biol. 12: 1079--1093.\par
Broennimann, O. et al. 2012. Measuring ecological niche overlap from
occurrence and spatial environmental data. - Glob. Ecol. Biogeogr. 21:
481--497.\par
Brown, J. H. 1984. On the Relationship between Abundance and
Distribution of Species. - Am. Nat. 124: 255--279.\par
Cáceres, M. De et al. 2015. Coupling a water balance model with forest
inventory data to predict drought stress: the role of forest structural
changes vs.~climate changes. - Agric. For. Meteorol. 213: 77--90.\par
Cadotte, M. W. and Tucker, C. M. 2017. Should Environmental Filtering be
Abandoned? - Trends Ecol. Evol. 32: 429--437.\par
Carnicer, J. et al. 2011. Widespread crown condition decline, food web
disruption, and amplified tree mortality with increased climate
change-type drought. - Proc. Natl. Acad. Sci. U. S. A. 108: 1474--8.\par
Cassel, D. K. and Nielsen, D. R. 1986. Field capacity and available
water capacity. - In: Klute, A. (ed), Methods of Soil Analysis: Part
1---Physical and Mineralogical Methods. Agronomy m. Soil Science Society
of America, pp.~901--926.\par
Clark, J. D. et al. 1993. A Multivariate Model of Female Black Bear
Habitat Use for a Geographic Information System. - J. Wildl. Manage. 57:
519.\par
Colwell, R. K. and Rangel, T. F. 2009. Hutchinson's duality: the once
and future niche. - Proc. Natl. Acad. Sci. U. S. A. 106 Suppl 2:
19651--8.\par
Colwell, R. K. et al. 2008. Global Warming, Elevational Range Shifts,
and Lowland Biotic Attrition in the Wet Tropics. Science. 322:
258--261.\par
Cornwell, W. K. and Ackerly, D. D. 2009. Community Assembly and Shifts
in Plant Trait Distributions across an Environmental Gradient in Coastal
California. - Source Ecol. Monogr. 79.\par
Coumou, D. and Rahmstorf, S. 2012. A decade of weather extremes. - Nat.
Clim. Chang. 2: 491--496.\par
Csergő, A. M. et al. 2017. Less favourable climates constrain
demographic strategies in plants (J Gurevitch, Ed.). - Ecol. Lett. 20:
969--980.\par
D'Amen, M. et al. 2017. Spatial predictions at the community level: From
current approaches to future frameworks. - Biol. Rev.~92: 169--187.\par
Dallas, T. A. and Hastings, A. 2018. Habitat suitability estimated by
niche models is largely unrelated to species abundance. - Glob. Ecol.
Biogeogr. 27: 1448--1456.\par
Dallas, T. et al. 2017. Species are not most abundant in the centre of
their geographic range or climatic niche. - Ecol. Lett. 20:
1526--1533.\par
Davis, M. B. 1986. Climatic Instability, Time, Lags, and Community
Disequilibrium.: 269--284.\par
Davis, M. B. and Shaw, R. G. 2001. Range shifts and adaptive responses
to quaternary climate change. Science. 292: 673--679.\par
Davis, K. T. et al. 2019. Microclimatic buffering in forests of the
future: the role of local water balance. - Ecography (Cop.). 42:
1--11.\par
De Frenne, P. et al. 2013. Microclimate moderates plant responses to
macroclimate warming. - Proc. Natl. Acad. Sci. U. S. A. 110: 18561--5.
de la Riva, E. G. et al. 2016a. Leaf Mass per Area (LMA) and Its
Relationship with Leaf Structure and Anatomy in 34 Mediterranean Woody
Species along a Water Availability Gradient. - PLoS One 11:
e0148788.\par
de la Riva, E. G. et al. 2016. Disentangling the relative importance of
species occurrence, abundance and intraspecific variability in community
assembly: a trait-based approach at the whole-plant level in
Mediterranean forests. - Oikos 125: 354--363.\par
de la Riva, E. G. et al. 2017. A Multidimensional Functional Trait
Approach Reveals the Imprint of Environmental Stress in Mediterranean
Woody Communities. - Ecosystems 21: 248--262.\par
del Cacho, M. and Lloret, F. 2012. Resilience of Mediterranean shrubland
to a severe drought episode: The role of seed bank and seedling
emergence. - Plant Biol. 14: 458--466.\par
Devictor, V. et al. 2012. Differences in the climatic debts of birds and
butterflies at a continental scale. - Nat. Clim. Chang. 2: 121--124.\par
Diaz, S. et al. 1998. Plant functional traits and environmental filters
at a regional scale. - J. Veg. Sci. 9: 113--122.\par
Dormann, C. F. 2007. Promising the future? Global change projections of
species distributions. - Basic Appl. Ecol. 8: 387--397.\par
Duchaufour, P. 1970. Pédologie (M\& Paris, Ed.).\par
Duong, T. 2018. Package ``ks''. ks: Kernel Smoothing. in press.\par
Duong, T. and Hazelton, M. L. 2005. Cross-validation bandwidth matrices
for multivariate kernel density estimation. - Scand. J. Stat. 32:
485--506.\par
Easterling, D. R. et al. 2000. Climate Extremes: Observations, Modeling,
and Impacts. Science. 289: 2068--2074.\par
Elith, J. 2006. Novel methods improve prediction of species'
distributions from occurrence data: Supplement. - Ecography (Cop.). 29:
129--151.\par
Elith, J. and Leathwick, J. R. 2009. Species Distribution Models:
Ecological Explanation and Prediction Across Space and Time. - Annu.
Rev.~Ecol. Evol. Syst. 40: 677--697.\par
Elith, J. and Graham, C. H. 2009. Do they? How do they? WHY do they
differ? on finding reasons for differing performances of species
distribution models. - Ecography (Cop.). 32: 66--77.\par
Elith, J. et al. 2002. Mapping epistematic uncertainties and vague
concepts in predictions of species distribution. - Ecol. Modell. 157:
313--329.\par
Elith, J. et al. 2008. A working guide to boosted regression trees. - J.
Anim. Ecol. 77: 802--813.\par
Elith, J. et al. 2010. The art of modelling range-shifting species. -
Methods Ecol. Evol. 1: 330--342.\par
Elton, C. S. 1927. Animal ecology. - University of Chicago Press.\par
Esteve-Selma, M. A. et al. 2010. Effects of climatic change on the
distribution and conservation of Mediterranean forests: The case of
Tetraclinis articulata in the Iberian Peninsula. - Biodivers. Conserv.
19: 3809--3825.\par
Esteve-Selma, M. A. et al. 2015. Cambio climático y biodiversidad en el
contexto de la Región de Murcia. - In: Consejería de Agua Agricultura y
Medio Ambiente (ed), Cambio climático en la Región de Murcia. Evaluación
basada en indicadores. pp.~105--132.\par
Fick, S. E. and Hijmans, R. J. 2017. WorldClim 2: new 1-km spatial
resolution climate surfaces for global land areas. - Int. J. Climatol.
37: 4302--4315.\par
Fielding, A. H. and Bell, J. F. 1997. A review of methods for the
assessment of prediction errors in conservation presence/absence models.
- Environ. Conserv. 24: 38--49.\par
Franklin, J. 2010. Mapping species distributions. Spatial inference and
prediction. - Cambridge University Press.\par
Franklin, J. et al. 2013. Modeling plant species distributions under
future climates: how fine scale do climate projections need to be? -
Glob. Chang. Biol. 19: 473--483.\par
Franklin, J. et al. 2016. Global change and terrestrial plant community
dynamics. - Proc. Natl. Acad. Sci. 113: 3725--3734.\par
Franks, S. J. et al. 2014. Evolutionary and plastic responses to climate
change in terrestrial plant populations. - Evol. Appl. 7: 123--139.\par
Freckleton, R. P. et al. 2002. Phylogenetic Analysis and Comparative
Data: 160: 712--726.\par
Fridley, J. 2009. Downscaling Climate over Complex Terrain: High
Finescale (\textless{}1000 m) Spatial Variation of Near-Ground
Temperatures in a Montane Forested Landscape (Great Smoky Mountains)*. -
J. Appl. Meteorol. Climatol. 48: 1033--1049.\par
Fridley, J. D. et al. 2011. Soil heterogeneity buffers community
response to climate change in species-rich grassland. - Glob. Chang.
Biol. 17: 2002--2011.\par
Galiano, L. et al. 2010. Drought-Induced Multifactor Decline of Scots
Pine in the Pyrenees and Potential Vegetation Change by the Expansion of
Co-occurring Oak Species. - Ecosystems 13: 978--991.\par
Gause, G. F. 1934. The struggle for existence (Williams \& Wilkin,
Ed..\par
Gaüzère, P. et al. 2018. Empirical Predictability of Community Responses
to Climate Change. - Front. Ecol. Evol. 6: 186.\par
GBIF.org (18 January 2019), GBIF Occurrence Download
\url{https://doi.org/10.15468/dl.er7c3e} \par
Gee, G. W. and Bauder, J. W. 1986. Particle size analysis. - In: Klute,
A. (ed), Methods of Soil Analysis: Part 1---Physical and Mineralogical
Methods. Soil Science Society of America, pp.~383--411.\par
Geiger, R. et al. 1995. The Climate Near the Ground. - Vieweg+Teubner
Verlag.\par
Giorgi, F. and Lionello, P. 2008. Climate change projections for the
Mediterranean region. - Glob. Planet. Change 63: 90--104.\par
Gotelli, N. J. et al. 2010. Macroecological signals of species
interactions in the Danish avifauna. - Proc. Natl. Acad. Sci. U. S. A.
107: 5030--5.\par
Graae, B. J. et al. 2018. Stay or go -- how topographic complexity
influences alpine plant population and community responses to climate
change. - Perspect. Plant Ecol. Evol. Syst. 30: 41--50.\par
Graham, C. H. et al. 2004. Integrating phylogenetics and environmental
niche models to explore speciation mechanisms in dendrobatid frogs. -
Evolution 58: 1781--93.\par
Grant, P. R. et al. 2016. Evolution caused by extreme events. - Philos.
Trans. R. Soc. Lond. B. Biol. Sci. 372: 20160146.\par
Greenwood, S. et al. 2017. Tree mortality across biomes is promoted by
drought intensity, lower wood density and higher specific leaf area. -
Ecol. Lett. 20: 539--553.\par
Grinnell, J. 1917. The Niche-Relationships of the California Thrasher. -
Auk 34: 427--433.\par
Guiot, J. and Cramer, W. 2016. Climate change: The 2015 Paris Agreement
thresholds and Mediterranean basin ecosystems. Science. 354:
4528--4532.\par
Guisan, A. and Zimmermann, N. E. 2000. Predictive habitat distribution
models in ecology. - Ecol. Modell. 135: 147--186.\par
Guisan, A. and Thuiller, W. 2005. Predicting species distribution:
Offering more than simple habitat models. - Ecol. Lett. 8:
993--1009.\par
Guisan, A. et al. 2013. Predicting species distributions for
conservation decisions (H Arita, Ed.). - Ecol. Lett. 16: 1424--1435.\par
Guisan, A. et al. 2014. Unifying niche shift studies: insights from
biological invasions. - Trends Ecol. Evol. 29: 260--269.\par
Guisan, A. et al. 2017. Habitat Suitability and Distribution Models
(Cambridge University Press, Ed.). - Cambridge University Press.\par
Guisan, A. et al. 2019. Scaling the linkage between environmental niches
and functional traits for improved spatial predictions of biological
communities (A Hampe, Ed.). - Glob. Ecol. Biogeogr.: geb.12967.\par
Hamerlynck, E. P. and McAuliffe, J. R. 2008. Soil-dependent canopy
die-back and plant mortality in two Mojave Desert shrubs. - J. Arid
Environ. 72: 1793--1802.\par
Hampe, A. and Petit, R. J. 2005. Conserving biodiversity under climate
change: the rear edge matters. - Ecol. Lett. 8: 461--467.\par
Hanley, A. J. and McNeil, J. B. 1982. The Meaning and Use of the Area
under a Receiver Operating Characteristic (ROC) Curve. - Radiology 143:
29--36.\par
Hannah, L. et al. 2007. Protected area needs in a changing climate. -
Front. Ecol. Environ. 5: 131--138.\par
Hanski, I. 1999. Metapopulation ecology. - Oxford University Press.\par
Hewitt, G. 2000. The genetic legacy of the Quaternary ice ages. - Nature
405\par
Hijmans, R. J. et al. 2005. Very high resolution interpolated climate
surfaces for global land areas. - Int. J. Climatol. 25: 1965--1978.\par
Hijmans, R. J. et al. 2011. Package `dismo .' - October: 55.\par
Hijmans, A. R. J. et al. 2016. Package `dismo' Species Distribution
Modeling. -
\url{https://cran.r-project.org/web/packages/dismo/dismo.pdf}: (accesed
11.01.2016).\par
HilleRisLambers, J. et al. 2012. Rethinking Community Assembly through
the Lens of Coexistence Theory. - Annu. Rev.~Ecol. Evol. Syst. 43:
227--248.\par
Hirzel, A. H. et al. 2006. Evaluating the ability of habitat suitability
models to predict species presences. - Ecol. Modell. 199: 142--152.\par
Holt, R. D. 2009. Bringing the Hutchinsonian niche into the 21st
century: Ecological and evolutionary perspectives. - Proc. Natl. Acad.
Sci. U. S. A. 106: 19659--19665.\par
Holt, R. D. et al. 2005. Theoretical models of species' borders: single
species approaches. - Oikos 108: 18--27.\par
Hubbell, S. P. 2001. The unified neutral theory of biodiversity and
biogeography. - Princeton University Press.\par
Huberty, C. J. 1994. Applied discriminant analysis. - Wiley.\par
Huntley, B. and Webb, T. 1989. Migration: Species' Response to Climatic
Variations Caused by Changes in the Earth's Orbit. - J. Biogeogr. 16:
5.\par
Hutchinson, G. E. 1957. Concluding Remarks - The Demographic Symposium
as a Heterogeneous Unstable Population. 53: 415--427.\par
Hutchinson, G. E. 1978. An introduction to population ecology. - Yale
University Press.\par
IPCC Working Group 1 2014. IPCC Fifth Assessment Report (AR5) - The
physical science basis (VB and PMM (eds. . Stocker, T.F., D. Qin, G.-K.
Plattner, M. Tignor, S.K. Allen, J. Boschung, A. Nauels, Y. Xia,
Ed.).\par
IUSS Working Group WRB 2015. World Reference Base for Soil Resources
2014, update 2015. International soil classification system for naming
soils and creating legends for soil maps. - World Soil Resour. Reports
No. 106.\par
Jackson, S. T. 2009. Introduction. - In: (editor), S. T. J. (ed), Essay
on the Geography of Plants. The University of Chicago Press,
pp.~1--52.\par
Jackson, S. T. and Sax, D. F. 2010. Balancing biodiversity in a changing
environment: extinction debt, immigration credit and species turnover. -
Trends Ecol. Evol. 25: 153--160.\par
Jaime, L. et al. 2019. Scots pine (Pinus sylvestris L.) mortality is
explained by the climatic suitability of both host tree and bark beetle
populations. - For. Ecol. Manage. 448: 119--129.\par
Jentsch, A. et al. 2007. A new generation of events , not trends
experiments. - Front. Ecol. Environ. 5: 365--374.\par
Joppa, L. N. et al. 2013. Troubling trends in scientific software use. -
Science (80-. ). 340: 814--815.\par
Jump, A. S. and Woodward, F. I. 2003. Seed production and population
density decline approaching the range-edge of Cirsium species. - New
Phytol. 160: 349--358.\par
Jump, A. S. et al. 2006. Rapid climate change-related growth decline at
the southern range edge of Fagus sylvatica. - Glob. Chang. Biol. 12:
2163--2174.\par
Jump, A. S. et al. 2009. The altitude-for-latitude disparity in the
range retractions of woody species. - Trends Ecol. Evol. 24:
694--701.\par
Karger, D. N. et al. 2017. Climatologies at high resolution for the
earth's land surface areas. - Sci. Data 4: 170122.\par
Kearney, M. 2006. Habitat , environment and niche: what are we
modelling? - Oikos 115: 186--191.\par
Kearney, M. and Porter, W. 2009. Mechanistic niche modelling: Combining
physiological and spatial data to predict species' ranges. - Ecol. Lett.
12: 334--350.\par
Keddy, P. A. 1992. Assembly and response rules: two goals for predictive
community ecology. - J. Veg. Sci. 3: 157--164.\par
Kramer, P. J. and Boyer, J. S. 1995. Water relations of plants and
soils. - Academic Press.\par
Kreuzwieser, J. and Gessler, A. 2010. Global climate change and tree
nutrition: Influence of water availability. - Tree Physiol. 30:
1221--1234.\par
Kreyling, J. et al. 2011. Stochastic trajectories of succession
initiated by extreme climatic events. - Ecol. Lett. 14: 758--764.\par
Kruckeberg, A. R. 2002. Geology and plant life: the effects of landforms
and rock types on plants. - University of Washington Press.\par
Kuussaari, M. et al. 2009. Extinction debt: a challenge for biodiversity
conservation. - Trends Ecol. Evol. 24: 564--571.\par
Kuznetsova, A. et al. 2017. lmerTest Package: Tests in Linear Mixed
Effects Models. - J. Stat. Softw. 82: 1--26.\par
Lázaro, R. et al. 2001. Analysis of a 30-year rainfall record
(1967--1997) in semi--arid SE Spain for implications on vegetation. - J.
Arid Environ. 48: 373--395.\par
Leathwick, J. R. 1998. Are New Zealand's Nothofagus species in
equilibrium with their environment? - J. Veg. Sci. 9: 719--732.\par
Lembrechts, J. J. et al. 2019a. Comparing temperature data sources for
use in species distribution models: From in‐situ logging to remote
sensing. - Glob. Ecol. Biogeogr.: geb.12974.\par
Lembrechts, J. J. et al. 2019b. Incorporating microclimate into species
distribution models. - Ecography (Cop.). 42: 1267--1279.\par
Lenoir, J. and Svenning, J.-C. 2015. Climate-related range shifts - a
global multidimensional synthesis and new research directions. -
Ecography (Cop.). 38: 15--28.\par
Lenoir, J. et al. 2008. A Significant Upward Shift in Plant Species
Optimum Elevation During the 20th Century. Science. 320: 1768--1771.\par
Lenoir, J. et al. 2013. Local temperatures inferred from plant
communities suggest strong spatial buffering of climate warming across
Northern Europe. - Glob. Chang. Biol. 19: 1470--1481.\par
Lenth, R. V. 2016. Least-Squares Means: The R Package lsmeans. - J.
Stat. Softw. 69: 1--33.\par
Lesica, P. and Crone, E. E. 2016. Arctic and boreal plant species
decline at their southern range limits in the Rocky Mountains. - Ecol.
Lett.: 166--174.\par
Lévesque, M. et al. 2016. Soil nutrients influence growth response of
temperate tree species to drought (R Jones, Ed.). - J. Ecol. 104:
377--387.\par
Li, Y. and Shipley, B. 2018. Community divergence and convergence along
experimental gradients of stress and disturbance. - Ecology 99:
775--781.\par
Li, Y. et al. 2018. Habitat filtering determines the functional niche
occupancy of plant communities worldwide. - J. Ecol. 106:
1001--1009.\par
Lloret, F. and Granzow-de la Cerda, I. 2013. Plant competition and
facilitation after extreme drought episodes in Mediterranean shrubland:
Does damage to vegetation cover trigger replacement by juniper woodland?
- J. Veg. Sci. 24: 1020--1032.\par
Lloret, F. and García, C. 2016. Inbreeding and neighbouring vegetation
drive drought-induced die-off within juniper populations. - Funct. Ecol.
30: 1696--1704.\par
Lloret, F. and Kitzberger, T. 2018. Historical and event-based
bioclimatic suitability predicts regional forest vulnerability to
compound effects of severe drought and bark beetle infestation. - Glob.
Chang. Biol. in press.\par
Lloret, F. et al. 2012. Extreme climatic events and vegetation: The role
of stabilizing processes. - Glob. Chang. Biol. 18: 797--805.\par
Lloret, F. et al. 2016. Climatic events inducing die-off in
Mediterranean shrublands: are species responses related to their
functional traits? - Oecologia 180: 961--973.\par
Lomolino, M. V. et al. 2004. Foundations of biogeography: classic papers
with commentaries. - University of Chicago Press.\par
Macarthur, R. and Levins, R. 1967. The Limiting Similarity, Convergence,
and Divergence of Coexisting Species. - Am. Nat. 101: 377--385.\par
Maestre, F. T. and Cortina, J. 2002. Spatial patterns of surface soil
properties and vegetation in a Mediterranean semi-arid steppe. - Plant
Soil 241: 279--291.\par
Maggini, R. et al. 2011. Are Swiss birds tracking climate change?:
Detecting elevational shifts using response curve shapes. - Ecol.
Modell. 222: 21--32.\par
Maguire, B. 1973. Niche Response Structure and the Analytical Potentials
of Its Relationship to the Habitat. - Am. Nat. 107: 213--246.\par
Martinez-Meyer, E. et al. 2013. Ecological niche structure and rangewide
abundance patterns of species. - Biol. Lett. 9: 20120637--20120637.\par
Martínez-Vilalta, J. and Lloret, F. 2016. Drought-induced vegetation
shifts in terrestrial ecosystems: The key role of regeneration dynamics.
- Glob. Planet. Change 144: 94--108.\par
Mauri, A. et al. 2016. Pinus halepensis and Pinus brutia in Europe:
distribution, habitat, usage and threats. - In: European Atlas of Forest
Tree Species. in press.\par
Mauri, A. et al. 2017. EU-Forest, a high-resolution tree occurrence
dataset for Europe. - Sci. Data in press.\par
Mcdowell, N. et al. 2008. Mechanisms of Plant Survival and Mortality
during Drought: Why Do Some Plants Survive while Others Succumb to
Drought? - New Phytol. 178: 719--739.\par
McDowell, N. et al. 2019. Mechanisms of a coniferous woodland
persistence under drought and heat. - Environ. Res. Lett. in press.\par
Mellert, K. H. et al. 2011. Hypothesis-driven species distribution
models for tree species in the Bavarian Alps. - J. Veg. Sci. 22:
635--646.\par
Merow, C. et al. 2013. A practical guide to MaxEnt for modeling species'
distributions: what it does, and why inputs and settings matter. -
Ecography (Cop.). 36: 1058--1069.\par
Merow, C. et al. 2014. What do we gain from simplicity versus complexity
in species distribution models? - Ecography (Cop.). 37: 1267--1281.\par
Miriti, M. N. et al. 2007. Episodic death across species of desert
shrubs. - Ecology 88: 32--36.\par
Morueta-Holme, N. and Svenning, J.-C. 2018. Geography of Plants in the
New World: Humboldt's Relevance in the Age of Big Data. - Ann. Missouri
Bot. Gard. 103: 315--329.\par
Mouillot, F. et al. 2002. Simulating climate change impacts on fire
frequency and vegetation dynamics in a Mediterranean ecosystem. - Glob.
Chang. Biol. 8: 423--437.\par
Murphy, H. T. et al. 2006. Distribution of abundance across the range in
eastern North American trees. - Glob. Ecol. Biogeogr. 15: 63--71.\par
Niehaus, A. C. et al. 2012. Predicting the physiological performance of
ectotherms in fluctuating thermal environments. - J. Exp. Biol. 215:
694--701.\par
Ninyerola, M. et al. 2000. A methodological approach of climatological
modelling of air temperature and precipitation through GIS techniques. -
Int. J. Climatol. 20: 1823--1841.\par
Ninyerola, M. et al. 2007. Monthly precipitation mapping of the Iberian
Peninsula using spatial interpolation tools implemented in a Geographic
Information System. - Theor. Appl. Climatol. 89: 195--209.\par
Osorio‐Olvera, L. et al. 2019. On population abundance and niche
structure. - Ecography (Cop.). 42: 1415--1425.\par
Parmesan, C. 2006. Ecological and Evolutionary Responses to Recent
Climate Change. - Annu. Rev.~Ecol. Evol. Syst. 37: 637--669.\par
Parmesan, C. and Yohe, G. 2003. A globally coherent fingerprint of
climate change impacts across natural systems. - Nature 421: 37--42.\par
Pausas, J. G. and Bond, W. J. 2018. Humboldt and the reinvention of
nature. - J. Ecol. in press.\par
Pearman, P. B. et al. 2008. Niche dynamics in space and time. - Trends
Ecol. Evol. 23: 149--158.\par
Pearson, R. G. and Dawson, T. P. 2003. Predicting the Impacts of Climate
Change on the Distribution of Species: Are Bioclimate Envelope Models
Useful? - Glob. Ecol. Biogeogr. 12: 361--371.\par
Pearson, R. G. et al. 2006. Model-based uncertainty in species range
prediction. - J. Biogeogr. 33: 1704--1711.\par
Pearson, D. E. et al. 2018. Community Assembly Theory as a Framework for
Biological Invasions. - Trends Ecol. Evol. 33: 313--325.\par
Peñuelas, J. et al. 2001. Severe drought effects on mediterranean woody
flora in Spain. - For. Sci. 47: 214--218.\par
Pérez-Ramos, I. M. et al. 2017. Climate variability and community
stability in Mediterranean shrublands: the role of functional diversity
and soil environment. - J. Ecol. 105: 1335--1346.\par
Pérez Navarro, M. Á. et al. 2018. Climatic Suitability Derived from
Species Distribution Models Captures Community Responses to an Extreme
Drought Episode. - Ecosystems: 1--14.\par
Peterson, A. T. 2011. Ecological Niches and Geographic Distributions
(Princeton University Press, Ed.).\par
Peterson, A. T. and Vieglais, D. A. 2001. Predicting Species Invasions
Using Ecological Niche Modeling: New Approaches from Bioinformatics
Attack a Pressing Problem. - Bioscience 51: 363--371.\par
Petitpierre, B. et al. 2012. Climatic Niche Shifts Are Rare Among
Terrestrial Plant Invaders. Science. 335: 1344--1348.\par
Phillips, S. J. and Dudík, M. 2008a. Modeling of species distribution
with Maxent: new extensions and a comprehensive evalutation. - Ecography
(Cop.). 31: 161--175.\par
Phillips, S. J. and Dudík, M. 2008b. Modeling of species distributions
with Maxent: new extensions and a comprehensive evaluation. - Ecography
(Cop.). 31: 161--175.\par
Piedallu, C. et al. 2013. Soil water balance performs better than
climatic water variables in tree species distribution modelling. - Glob.
Ecol. Biogeogr. 22: 470--482.\par
Pironon, S. et al. 2015. Do geographic, climatic or historical ranges
differentiate the performance of central versus peripheral populations?
- Glob. Ecol. Biogeogr. 24: 611--620.\par
Pironon, S. et al. 2016. Geographic variation in genetic and demographic
performance: new insights from an old biogeographical paradigm. - Biol.
Rev.: 000--000.\par
Pironon, S. et al. 2017. The `Hutchinsonian niche' as an assemblage of
demographic niches: implications for species geographic ranges. -
Ecography (Cop.). in press.\par
Porporato, A. et al. 2004. Soil Water Balance and Ecosystem Response to
Climate Change. - Am. Nat. 164: 625--632.\par
Prentice, I. C. et al. 1992. A Global Biome Model Based on Plant
Physiology and Dominance, Soil Properties and Climate. - J. Biogeogr.
19: 117.\par
Pulliam, H. R. 1988. Sources, Sinks, and Population Regulation. - Am.
Nat. 132: 652--661.\par
Pulliam, H. R. 2000. On the relationship between niche and distribution.
- Ecol. Lett. 3: 349--361.\par
Randin, C. F. et al. 2009. Climate change and plant distribution: local
models predict high-elevation persistence. - Glob. Chang. Biol. 15:
1557--1569.\par
Ricklefs, R. E. 2008. Disintegration of the ecological community. - Am.
Nat. 172: 741--750.\par
Rivas-Martínez, S. et al. 2011. Worldwide bioclimatic classification
system. - Glob. Geobot. 1: 1--638.\par
Rivas-Martínez, S. et al. 2017. Bioclimatology of the Iberian Peninsula
and the Balearic Islands. - In: Springer, Cham, pp.~29--80.\par
Sangüesa-Barreda, G. et al. 2018. Delineating limits: Confronting
predicted climatic suitability to field performance in mistletoe
populations. - J. Ecol. 106: 2218--2229.\par
Sapes, G. et al. 2017. Species climatic niche explains drought-induced
die-off in a Mediterranean woody community. - Ecosphere 8: e01833.\par
Schurr, F. M. et al. 2012. How to understand species' niches and range
dynamics: A demographic research agenda for biogeography. - J. Biogeogr.
39: 2146--2162.\par
Segurado, P. and Araújo, M. B. 2004. An evaluation of methods for
modelling species distributions. - J. Biogeogr. 31: 1555--1568.\par
Serra-Diaz, J. M. et al. 2013. Geographical patterns of congruence and
incongruence between correlative species distribution models and a
process-based ecophysiological growth model. - J. Biogeogr. 40:
1928--19338.\par
Sexton, J. P. et al. 2009. Evolution and Ecology of Species Range
Limits. - Annu. Rev.~Ecol. Evol. Syst 40: 415--36.\par
Sexton, J. P. et al. 2014. Genetic isolation by environment or distance:
which pattern of gene flow is most common? - Evolution. 68: 1--15.\par
Shantz, H. L. 1927. Drought Resistance and Soil Moisture. - Ecology 8:
145--157.\par
Sheffield, J. and Wood, E. F. 2008. Projected changes in drought
occurrence under future global warming from multi-model , multi-scenario
, IPCC AR4 simulations. - Clim. Dyn. 31: 79--105.\par
Simpson, A. H. et al. 2016. Soil--climate interactions explain variation
in foliar, stem, root and reproductive traits across temperate forests.
- Glob. Ecol. Biogeogr. 25: 964--978.\par
Soberón, J. 2007. Grinnellian and Eltonian niches and geographic
distributions of species. - Ecol. Lett. 10: 1115--1123.\par
Soberón, J. and Nakamura, M. 2009. Niches and distributional areas:
Concepts, methods, and assumptions. - Proc. Natl. Acad. Sci. 106:
19645--19650.\par
Solarik, K. A. et al. 2018. Local adaptation of trees at the range
margins impacts range shifts in the face of climate change. - Glob.
Ecol. Biogeogr. 27: 1507--1519.\par
Suggitt, A. J. et al. 2011. Habitat microclimates drive fine-scale
variation in extreme temperatures. - Oikos 120: 1--8.\par
Svenning, J.-C. and Skov, F. 2004. Limited filling of the potential
range in European tree species. - Ecol. Lett. 7: 565--573.\par
Svenning, J.-C. and Skov, F. 2007. Could the tree diversity pattern in
Europe be generated by postglacial dispersal limitation? - Ecol. Lett.
10: 453--460.\par
Svenning, J. C. and Sandel, B. 2013. Disequilibrium vegetation dynamics
under future climate change. - Am. J. Bot. 100: 1266--1286.\par
Takahashi, K. and Tanaka, S. 2016. Relative importance of habitat
filtering and limiting similarity on species assemblages of alpine and
subalpine plant communities. - J. Plant Res. 129: 1041--1049.\par
Thibault, K. M. and Brown, J. H. 2008. Impact of an extreme climatic
event on community assembly. - Proc. Natl. Acad. Sci. 105:
3410--3415.\par
Thomas, C. D. et al. 2004. Extinction risk from climate change. - Nature
427: 145--148.\par
Thornthwaite, C. W. and Mather, J. R. 1957. Instructions and tables for
computing potential evapotranspiration and the water balance,. - Publ.
Climatol. 10: 185--311.\par
Thuiller, W. 2004. Patterns and uncertainties of species' range shifts
under climate change. - Glob. Chang. Biol. 10: 2020--2027.\par
Thuiller, W. 2013. On the importance of edaphic variables to predict
plant species distributions - limits and prospects. - J. Veg. Sci. 24:
591--592.\par
Thuiller, W. et al. 2005a. Niche properties and geographical extent as
predictors of species sensitivity to climate change. - Glob. Ecol.
Biogeogr. 14: 347--357.\par
Thuiller, W. et al. 2005b. Climate change threats to plant diversity in
Europe. - Proc. Natl. Acad. Sci. U. S. A. 102 (23): 8245-8250.\par
Thuiller, W. et al. 2010. Variation in habitat suitability does not
always relate to variation in species' plant functional traits. - Biol.
Lett. 6: 120--123.\par
Thuiller, W. et al. 2014. Does probability of occurrence relate to
population dynamics? - Ecography (Cop.). 37: 1155--1166.\par
Tulloch, A. I. T. et al. 2016. Conservation planners tend to ignore
improved accuracy of modelled species distributions to focus on multiple
threats and ecological processes. - Biol. Conserv. 199: 157--171.\par
Ulrich, W. et al. 2014. Climate and soil attributes determine plant
species turnover in global drylands. - J. Biogeogr. 41: 2307--2319.\par
Valladares, F. et al. 2004. CAPÍTULO 6 Estrés hídrico: ecofisiología y
escalas de la sequía. - In: Valladares, F. (ed), Ecologia del bosque
mediterráneo en un mundo cambiante. Ministerio. pp.~163--190.\par
Valladares, F. et al. 2008. Functional traits and phylogeny: What is the
main ecological process determining species assemblage in roadside plant
communities? - J. Veg. Sci. 19: 381--392.\par
Valladares, F. et al. 2014a. Global change and Mediterranean forests:
current impacts and potential responses. - In: Forests and Global
Change. pp.~47--75.\par
Valladares, F. et al. 2014b. The effects of phenotypic plasticity and
local adaptation on forecasts of species range shifts under climate
change. - Ecol. Lett. 17: 1351--1364.\par
van der Maaten, E. et al. 2017. Species distribution models predict
temporal but not spatial variation in forest growth. - Ecol. Evol. 7:
2585--2594.\par
van Mantgem, P. J. et al. 2009. Widespread Increase of Tree Mortality
Rates in the Western United States. Science. 323: 521--524.\par
VanDerWal, J. et al. 2009. Abundance and the environmental niche:
environmental suitability estimated from niche models predicts the upper
limit of local abundance. - Am. Nat. 174: 282--91.\par
Vellend, M. 2010. Conceptual synthesis in community ecology. - Q.
Rev.~Biol. 85: 183--206.\par
Vicente-Serrano, S. M. et al. 2013. Response of vegetation to drought
time-scales across global land biomes. - Proc. Natl. Acad. Sci. U. S. A.
110: 52--57.\par
Volterra, V. 1926. Fluctuations in the Abundance of a Species considered
Mathematically1. - Nature 118: 558--560.\par
von Humboldt, A. and Bonpland, A. 1807 (2009). Essay on the geography of
plants. Reprint, translated by Sylvie Romanowski, edited with an
introduction by S. T. Jackson. The University of Chicago Press.\par
Walther, G.-R. et al. 2009. Alien species in a warmer world: risks and
opportunities. - Trends Ecol. Evol. 24: 686--693.\par
Wan, Z. 2008. New refinements and validation of the MODIS Land-Surface
Temperature/Emissivity products. - Remote Sens. Environ. 112:
59--74.\par
Wan, Z. et al. 2015. MOD11C2 MODIS/Terra Land Surface
Temperature/Emissivity 8‐Day L3 Global 0.05° CMG V006 {[}Data set{]}. In
NASA EOSDIS LP DAAC.\par
Wang, J. and Maintainer, L. S. C. 2016. Package ``MixRF.'' in press.\par
Webb, T. 1986. Is vegetation in equilibrium with climate? How to
interpret late-Quaternary pollen data. - Vegetatio 67: 75--91.\par
Webb, C. O. and Donoghue, M. J. 2005. Phylomatic: Tree assembly for
applied phylogenetics. - Mol. Ecol. Notes 5: 181--183.\par
Webb III, T. 1992. Global Changes During the Last 3 Million Years:
Climatic Controls and Biotic Responses. - Annu. Rev.~eclogy Syst. 23:
141--173.\par
Weber, M. M. et al. 2016. Is there a correlation between abundance and
environmental suitability derived from ecological niche modelling? A
meta-analysis. - Ecography (Cop.). 40: 817--828.\par
Weiher, E. et al. 2011. Advances, challenges and a developing synthesis
of ecological community assembly theory. - Philos. Trans. R. Soc. B
Biol. Sci. 366: 2403--2413.\par
Wessels, K. J. et al. 1998. An evaluation of the gradsect biological
survey method. - Biodivers. Conserv. 7: 1093--1121.\par
Wiens, J. J. and Graham, C. H. 2005. Niche Conservatism: Integrating
Evolution, Ecology, and Conservation Biology. - Annu. Rev.~Ecol. Evol.
Syst. 36: 519--539.\par
Wiens, J. A. et al. 2009. Niches, models, and climate change: Assessing
the assumptions and uncertainties. - Proc. Natl. Acad. Sci. 106:
19729--19736.\par
Woodward, F. I. 1987. Climate and plant distribution. - Cambridge
University Press.\par
Wright, J. W. et al. 2006. Experimental verification of ecological niche
modeling in a heterogeneous environment. - Ecology 87: 2433--2439.\par
Zavaleta, E. S. et al. 2003. Grassland Responses To Three Years of
Elevated Temperature, Co 2 , Precipitation, and N Deposition. - Ecol.
Monogr. 73: 585--604.\par
Zimmermann, N. E. et al. 2009. Climatic extremes improve predictions of
spatial patterns of tree species. - Proc. Natl. Acad. Sci. U. S. A. 106
Suppl: 19723--8.\par
Zunzunegui, M. et al. 2005. To live or to survive in Doñana dunes:
Adaptive responses of woody species under a Mediterranean climate. -
Plant Soil 273: 77--89.\par

\hypertarget{refs}{}
\hypertarget{ref-Allen2015}{}
Allen, C. D., Breshears, D. D., \& McDowell, N. G. (2015). On
underestimation of global vulnerability to tree mortality and forest
die-off from hotter drought in the Anthropocene. \emph{Ecosphere},
\emph{6}(8), 1--55. \url{http://doi.org/10.1890/ES15-00203.1}

\hypertarget{ref-Barrett1995}{}
Barrett, D. J., Hatton, T. J., Ash, J. E., \& Ball, M. C. (1995).
Evaluation of the heat pulse velocity technique for measurement of sap
flow In rainforest and eucalypt forest species of south-eastern
Australia, (1), 463--469.

\hypertarget{ref-Barton2017}{}
Barton, K. (2017). \emph{MuMIn: Multi-Model Inference}. Retrieved from
\url{https://cran.r-project.org/package=MuMIn}

\hypertarget{ref-Bates2015}{}
Bates, D., Mächler, M., Bolker, B., \& Walker, S. (2015). Fitting Linear
Mixed-Effects Models Using \{lme4\}. \emph{Journal of Statistical
Software}, \emph{67}(1), 1--48.
\url{http://doi.org/10.18637/jss.v067.i01}

\hypertarget{ref-Becker1998}{}
Becker, P. (1998). Limitations of a compensation heat pulse velocity
system at low sap flow: implications for measurements at night and in
shaded trees. \emph{Tree Physiology}, \emph{18}(3), 177--184.
\url{http://doi.org/10.1093/treephys/18.3.177}

\hypertarget{ref-Bleby2004}{}
Bleby, T. M., Burgess, S. S. O., \& Adams, M. A. (2004). A validation,
comparison and error analysis of two heat-pulse methods for measuring
sap flow in Eucalyptus marginata saplings. \emph{Functional Plant
Biology}, \emph{31}(6), 645--658. \url{http://doi.org/10.1071/FP04013}

\hypertarget{ref-Bleby2008}{}
Bleby, T., McElrone, A., \& Burgess, S. (2008). Limitations of the HRM:
great at low flow rates, but not yet up to speed? In \emph{7th sap flow
workshop}. Seville.

\hypertarget{ref-Braun1999}{}
Braun, P., \& Schmid, J. (1999). Sap flow measurements in grapevines (
Vitis vinifera L .) 2. Granier measurements. \emph{Plant and Soil},
\emph{215}, 47--55.

\hypertarget{ref-Burgess2001}{}
Burgess, S. S., Adams, M. a, Turner, N. C., Beverly, C. R., Ong, C. K.,
Khan, a a, \& Bleby, T. M. (2001). An improved heat pulse method to
measure low and reverse rates of sap flow in woody plants. \emph{Tree
Physiology}, \emph{21}(9), 589--598.
\url{http://doi.org/10.1093/treephys/21.9.589}

\hypertarget{ref-Bush2010}{}
Bush, S. E., Hultine, K. R., Sperry, J. S., Ehleringer, J. R., \&
Phillips, N. (2010). Calibration of thermal dissipation sap flow probes
for ring- and diffuse-porous trees. \emph{Tree Physiology},
\emph{30}(12), 1545--1554. \url{http://doi.org/10.1093/treephys/tpq096}

\hypertarget{ref-Cabibel1991}{}
Cabibel, B., \& F, D. (1991). Mesures thermiques des flux de sève dans
les troncs et les racines et fonctionnement hydriques des arbres. I.
\emph{I.}, (fig 4).

\hypertarget{ref-Cain2009}{}
Cain, R. (2009). \emph{The climatic significance of tropical forest
edges and their representation in global climate models} (PhD thesis).
Durham University. Retrieved from \url{http://etheses.dur.ac.uk/302/}

\hypertarget{ref-Castelan-Estrada2002}{}
Castelan-Estrada, M., Vivin, P., \& Gaudillière, J. P. (2002).
Allometric relationships to estimate seasonal above-ground vegetative
and reproductive biomass of Vitis vinifera L. \emph{Annals of Botany},
\emph{89}(4), 401--408. \url{http://doi.org/10.1093/aob/mcf059}

\hypertarget{ref-Caterina2014}{}
Caterina, G. L., Will, R. E., Turton, D. J., Wilson, D. S., \& Zou, C.
B. (2014). Water use of Juniperus virginiana trees encroached into mesic
prairies in Oklahoma, USA. \emph{Ecohydrology}, \emph{7}(4), 1124--1134.
\url{http://doi.org/10.1002/eco.1444}

\hypertarget{ref-Chen2012}{}
Chen, X., Miller, G. R., Rubin, Y., \& Baldocchi, D. D. (2012). A
statistical method for estimating wood thermal diffusivity and probe
geometry using in situ heat response curves from sap flow measurements.
\emph{Tree Physiology}, \emph{32}(12), 1458--1470.
\url{http://doi.org/10.1093/treephys/tps100}

\hypertarget{ref-Clearwater2009}{}
Clearwater, M. J., Luo, Z., Mazzeo, M., \& Dichio, B. (2009). An
external heat pulse method for measurement of sap flow through fruit
pedicels, leaf petioles and other small-diameter stems. \emph{Plant,
Cell and Environment}, \emph{32}(12), 1652--1663.
\url{http://doi.org/10.1111/j.1365-3040.2009.02026.x}

\hypertarget{ref-Clearwater1999}{}
Clearwater, M. J., Meinzer, F. C., Andrade, J. L., Goldstein, G., \&
Holbrook, N. M. (1999). Potential errors in measurement of nonuniform
sap flow using heat dissipation probes. \emph{Tree Physiology},
\emph{19}(Equation 4), 681--687.
\url{http://doi.org/10.1093/treephys/19.10.681}

\hypertarget{ref-Cohen1981}{}
Cohen, Y., Fuchs, M., \& Green, G. C. (1981). Improvement of the heat
pulse method for determining sap flow in trees. \emph{Plant, Cell and
Environment}, \emph{4}, 391--397.
\url{http://doi.org/10.1111/j.1365-3040.1981.tb02117.x}

\hypertarget{ref-Cermak2004}{}
Čermák, J., Kučera, J., \& Nadezhdina, N. (2004). Sap flow measurements
with some thermodynamic methods, flow integration within trees and
scaling up from sample trees to entire forest stands. \emph{Trees -
Structure and Function}, \emph{18}(5), 529--546.
\url{http://doi.org/10.1007/s00468-004-0339-6}

\hypertarget{ref-DeOliveiraReis2006}{}
de Oliveira Reis, F., Campostrini, E., Sousa, E. F. de, \& Silva, M. G.
e. (2006). Sap flow in papaya plants: Laboratory calibrations and
relationships with gas exchanges under field conditions. \emph{Scientia
Horticulturae}, \emph{110}(3), 254--259.
\url{http://doi.org/10.1016/j.scienta.2006.07.010}

\hypertarget{ref-Diawara1991}{}
Diawara, a, Loustau, D., \& Berbigier, P. (1991). Comparison of 2
Methods for Estimating the Evaporation of A Pinus-Pinaster (Ait) Stand -
Sap Flow and Energy-Balance with Sensible Heat-Flux Measurements by An
Eddy Covariance Method. \emph{Agricultural and Forest Meteorology},
\emph{54}(1), 49--66. \url{http://doi.org/10.1016/0168-1923(91)90040-W}

\hypertarget{ref-Do2002}{}
Do, F., \& Rocheteau, a. (2002). Influence of natural temperature
gradients on measurements of xylem sap flow with thermal dissipation
probes. 2. Advantages and calibration of a noncontinuous heating system.
\emph{Tree Physiology}, \emph{22}(9), 649--654.
\url{http://doi.org/10.1093/treephys/22.9.649}

\hypertarget{ref-Fan2018}{}
Fan, J., Guyot, A., Ostergaard, K. T., \& Lockington, D. A. (2018).
Effects of earlywood and latewood on sap flux density-based
transpiration estimates in conifers. \emph{Agricultural and Forest
Meteorology}, \emph{249}(November 2017), 264--274.
\url{http://doi.org/10.1016/j.agrformet.2017.11.006}

\hypertarget{ref-Fathi2014}{}
Fathi, L. (2014). \emph{Structural and mechanical properties of the wood
from coconut palms, oil palms and date palms} (PhD thesis). Hamburg.

\hypertarget{ref-Frank2015}{}
Frank, D. C., Poulter, B., Saurer, M., Esper, J., Huntingford, C.,
Helle, G., \ldots{} Weigl, M. (2015). Water-use efficiency and
transpiration across European forests during the Anthropocene.
\emph{Nature Climate Change}, \emph{5}(6), 579--583.
\url{http://doi.org/10.1038/nclimate2614}

\hypertarget{ref-Fuchs2017}{}
Fuchs, S., Leuschner, C., Link, R., Coners, H., \& Schuldt, B. (2017).
Calibration and comparison of thermal dissipation, heat ratio and heat
field deformation sap flow probes for diffuse-porous trees.
\emph{Agricultural and Forest Meteorology}, \emph{244-245}(June),
151--161. \url{http://doi.org/10.1016/j.agrformet.2017.04.003}

\hypertarget{ref-Granier1985}{}
Granier, A., \& Une, A. G. (1985). Une nouvelle m ´ ethode pour la
mesure du flux de s ` eve brute dans le tronc des arbres.

\hypertarget{ref-Green1988}{}
Green, S. R., \& Clothier, B. E. (1988). Water use of kiwifruit vines
and apple trees by the heat pulse technique. \emph{Journal of
Experimental Botany}, \emph{39}(198), 115--123.

\hypertarget{ref-Green2003}{}
Green, S., Clothier, B., \& Jardine, B. (2003). Theory and Practical
Application of Heat Pulse to Measure Sap Flow. \emph{Agriculture}.

\hypertarget{ref-Green2009}{}
Green, S., Clothier, B., \& Perie, E. (2009). A re-analysis of heat
pulse theory across a wide range of sap flows. \emph{Acta
Horticulturae}, \emph{846}, 95--104.

\hypertarget{ref-Hanssens2013}{}
Hanssens, J., De Swaef, T., Nadezhdina, N., \& Steppe, K. (2013).
Measurement of sap flow dynamics through the tomato peduncle using a
non-invasive sensor based on the heat field deformation method.
\emph{Acta Horticulturae}, \emph{991}, 409--416.
\url{http://doi.org/10.17660/ActaHortic.2013.991.50}

\hypertarget{ref-Hatton1990}{}
Hatton, T. (1990). Integration of sapflow velocity to estimate plant
water use. \emph{Tree Physiology}, \emph{6}(v), 201--209.
\url{http://doi.org/10.1093/treephys/6.2.201}

\hypertarget{ref-Hatton1995}{}
Hatton, T. J., Moore, S. J., \& Reece, P. H. (1995). Estimating stand
transpiration in a Eucalyptus populnea woodland with the heat pulse
method: measurement errors and sampling strategies. \emph{Tree
Physiology}, \emph{15}(4), 219--227.
\url{http://doi.org/10.1093/treephys/15.4.219}

\hypertarget{ref-Hernandez-Santana2015}{}
Hernandez-Santana, V., Hernandez-Hernandez, A., Vadeboncoeur, M. A., \&
Asbjornsen, H. (2015). Scaling from single-point sap velocity
measurements to stand transpiration in a multispecies deciduous forest:
uncertainty sources, stand structure effect, and future scenarios.
\emph{Canadian Journal of Forest Research}, \emph{45}(11), 1489--1497.
\url{http://doi.org/10.1139/cjfr-2015-0009}

\hypertarget{ref-Hogg1997}{}
Hogg, E. H., Black, T. A., Hartog, G. den, Neumann, H. H., Zimmermann,
R., Hurdle, P. A., \ldots{} Oren, R. (1997). A comparison of sap flow
and eddy fluxes of water vapor from a boreal deciduous forest.
\emph{Journal of Geophysical Research}, \emph{102}(D24), 28929.
\url{http://doi.org/10.1029/96JD03881}

\hypertarget{ref-Holtta2015}{}
Hölttä, T., Linkosalo, T., Riikonen, A., Sevanto, S., \& Nikinmaa, E.
(2015). An analysis of Granier sap flow method, its sensitivity to heat
storage and a new approach to improve its time dynamics.
\emph{Agricultural and Forest Meteorology}, \emph{211-212}, 2--12.
\url{http://doi.org/10.1016/j.agrformet.2015.05.005}

\hypertarget{ref-Hultine2010}{}
Hultine, K. R., Nagler, P. L., Morino, K., Bush, S. E., Burtch, K. G.,
Dennison, P. E., \ldots{} Ehleringer, J. R. (2010). Sap flux-scaled
transpiration by tamarisk (Tamarix spp.) before, during and after
episodic defoliation by the saltcedar leaf beetle (Diorhabda
carinulata). \emph{Agricultural and Forest Meteorology}, \emph{150}(11),
1467--1475. \url{http://doi.org/10.1016/j.agrformet.2010.07.009}

\hypertarget{ref-Jasechko2013}{}
Jasechko, S., Sharp, Z. D., Gibson, J. J., Birks, S. J., Yi, Y., \&
Fawcett, P. J. (2013). Terrestrial water fluxes dominated by
transpiration. \emph{Nature}, \emph{496}(7445), 347--350.
\url{http://doi.org/10.1038/nature11983}

\hypertarget{ref-Kempe2014}{}
Kempe, A., Lautenschl??ger, T., Lange, A., \& Neinhuis, C. (2014). How
to become a tree without wood - biomechanical analysis of the stem of
Carica papaya L. \emph{Plant Biology}, \emph{16}(1), 264--271.
\url{http://doi.org/10.1111/plb.12035}

\hypertarget{ref-Konings2017}{}
Konings, A. G., Williams, A. P., \& Gentine, P. (2017). Sensitivity of
grassland productivity to aridity controlled by stomatal and xylem
regulation. \emph{Nature Geoscience}, \emph{10}(4), 284--288.
\url{http://doi.org/10.1038/ngeo2903}

\hypertarget{ref-Kool2014}{}
Kool, D., Agam, N., Lazarovitch, N., Heitman, J. L., Sauer, T. J., \&
Ben-gal, a. (2014). A review of approaches for evapotranspiration
partitioning Author ' s personal copy. \emph{Agricultural and Forest
Meteorology}, \emph{184}, 56--70.
\url{http://doi.org/10.1016/j.agrformet.2013.09.003}

\hypertarget{ref-Lenth2016}{}
Lenth, R. V. (2016). Least-Squares Means: The \{R\} Package \{lsmeans\}.
\emph{Journal of Statistical Software}, \emph{69}(1), 1--33.
\url{http://doi.org/10.18637/jss.v069.i01}

\hypertarget{ref-Looker2016}{}
Looker, N., Martin, J., Jencso, K., \& Hu, J. (2016). Contribution of
sapwood traits to uncertainty in conifer sap flow as estimated with the
heat-ratio method. \emph{Agricultural and Forest Meteorology},
\emph{223}, 60--71. \url{http://doi.org/10.1016/j.agrformet.2016.03.014}

\hypertarget{ref-Lu2002}{}
Lu, P. (2002). Whole-plant water use of some tropical and subtropical
tree crops and its application in irrigation management., (575(Vol. 2)),
781--789.

\hypertarget{ref-Lu1998}{}
Lu, P., \& Chacko, E. (1998). Evaluation of Granier's sap flux in young
mango trees sensor. \emph{Agronomie}, \emph{18}(August 1997), 461--471.

\hypertarget{ref-Lu2004}{}
Lu, P., Urban, L., \& Zhao, P. (2004). Granier's thermal dissipation
probe (TDP) method for measuring sap flow in trees: theory and practice.
\emph{Acta Botanica Sinica}, \emph{46}(6), 631--646.

\hypertarget{ref-MacLean1941}{}
MacLean, J. (1941). Thermal Conductivity of wood. \emph{Heating, Piping
\& Air Conditioning}, \emph{13}(6), 380--391.

\hypertarget{ref-Maranon-jimenez2018}{}
Marañón-Jiménez, S., Bulcke, J. den, Piayda, A., Van Acker, J., Cuntz,
M., Rebmann, C., \& Steppe, K. (2018). X-ray computed microtomography
characterizes the wound effect that causes sap flow underestimation by
thermal dissipation sensors. \emph{Tree Physiology}, \emph{38}(2),
287--301. \url{http://doi.org/10.1093/treephys/tpx103}

\hypertarget{ref-Martinez-Vilalta2007}{}
Martínez-Vilalta, J., Korakaki, E., Vanderklein, D., \& Mencuccini, M.
(2007). Below-ground hydraulic conductance is a function of
environmental conditions and tree size in Scots pine. \emph{Functional
Ecology}, \emph{21}(6), 1072--1083.
\url{http://doi.org/10.1111/j.1365-2435.2007.01332.x}

\hypertarget{ref-McCulloh2007}{}
McCulloh, K. a, Winter, K., Meinzer, F. C., Garcia, M., Aranda, J., \&
Lachenbruch, B. (2007). A comparison of daily water use estimates
derived from constant-heat sap-flow probe values and gravimetric
measurements in pot-grown saplings. \emph{Tree Physiology},
\emph{27}(9), 1355--1360.
\url{http://doi.org/10.1093/treephys/27.9.1355}

\hypertarget{ref-Mitchell2009}{}
Mitchell, R. J., Irwin, R. E., Flanagan, R. J., \& Karron, J. D. (2009).
Ecology and evolution of plant-pollinator interactions. \emph{Annals of
Botany}, \emph{103}(9), 1355--1363.
\url{http://doi.org/10.1093/aob/mcp122}

\hypertarget{ref-Montague2006}{}
Montague, T., \& Kjelgren, R. (2006). Use of Thermal Dissipation Probes
to Estimate Water Loss of Containerized Landscape Trees. \emph{Journal
of Environmental Horticulture}, \emph{24}(2), 95--104.

\hypertarget{ref-Nadezhdina2018}{}
Nadezhdina, N. (2018). Revisiting the heat field deformation (HFD)
method for measuring sap flow. \emph{IForest}, \emph{11}(1), 118--130.
\url{http://doi.org/10.3832/ifor2381-011}

\hypertarget{ref-Nadezhdina1998}{}
Nadezhdina, N., Cermák, J., \& Nadezhdin, V. (1998). Heat field
deformation method for sap flow measurements. Zidlochovice, Czech
Republic: IUFRO Publications.

\hypertarget{ref-Nakagawa2013}{}
Nakagawa, S., \& Schielzeth, H. (2013). A general and simple method for
obtaining R2 from generalized linear mixed-effects models. \emph{Methods
in Ecology and Evolution}, \emph{4}(2), 133--142.
\url{http://doi.org/10.1111/j.2041-210x.2012.00261.x}

\hypertarget{ref-Novick2016}{}
Novick, K. A., Ficklin, D. L., Stoy, P. C., Williams, C. A., Bohrer, G.,
Oishi, A. C. C., \ldots{} Phillips, R. P. (2016). The increasing
importance of atmospheric demand for ecosystem water and carbon fluxes.
\emph{Nature Climate Change}, \emph{1}(September), 1--5.
\url{http://doi.org/10.1038/nclimate3114}

\hypertarget{ref-Oishi2016}{}
Oishi, A. C., Hawthorne, D. A., \& Oren, R. (2016). Baseliner: An
open-source, interactive tool for processing sap flux data from thermal
dissipation probes. \emph{SoftwareX}, \emph{5}, 139--143.
\url{http://doi.org/10.1016/j.softx.2016.07.003}

\hypertarget{ref-Oishi2008}{}
Oishi, A. C., Oren, R., \& Stoy, P. C. (2008). Estimating components of
forest evapotranspiration: A footprint approach for scaling sap flux
measurements. \emph{Agricultural and Forest Meteorology},
\emph{148}(11), 1719--1732.
\url{http://doi.org/10.1016/j.agrformet.2008.06.013}

\hypertarget{ref-Oliveras2001}{}
Oliveras, I., \& Llorens, P. (2001). Medium-term sap flux monitoring in
a Scots pine stand: analysis of the operability of the heat dissipation
method for hydrological purposes. \emph{Tree Physiology}, \emph{21}(7),
473--480. \url{http://doi.org/10.1093/treephys/21.7.473}

\hypertarget{ref-Paudel2013}{}
Paudel, I., Kanety, T., \& Cohen, S. (2013). Inactive xylem can explain
differences in calibration factors for thermal dissipation probe sap
flow measurements. \emph{Tree Physiology}, \emph{33}(9), 986--1001.
\url{http://doi.org/10.1093/treephys/tpt070}

\hypertarget{ref-Pearsall2014}{}
Pearsall, K. R., Williams, L. E., Castorani, S., Bleby, T. M., \&
Mcelrone, A. J. (2014). Evaluating the potential of a novel dual
heat-pulse sensor to measure volumetric water use in grapevines under a
range of flow conditions. \emph{Functional Plant Biology}, \emph{41}(8),
874--883. \url{http://doi.org/10.1071/FP13156}

\hypertarget{ref-Peters2018}{}
Peters, R. L., Fonti, P., Frank, D. C., Poyatos, R., Pappas, C., Kahmen,
A., \ldots{} Steppe, K. (2018). Quantification of uncertainties in tree
sap flow measured with the thermal dissipation method. \emph{Under
Review}. \url{http://doi.org/10.1111/nph.15241}

\hypertarget{ref-Poyatos2016}{}
Poyatos, R., Granda, V., Molowny-Horas, R., Mencuccini, M., Steppe, K.,
\& Martínez-Vilalta, J. (2016). SAPFLUXNET: Towards a global database of
sap flow measurements. \emph{Tree Physiology}, \emph{36}(12),
1449--1455. \url{http://doi.org/10.1093/treephys/tpw110}

\hypertarget{ref-RCoreTeam2017}{}
R Core Team. (2017). \emph{R: A Language and Environment for Statistical
Computing}. Vienna, Austria: R Foundation for Statistical Computing.
Retrieved from \url{https://www.r-project.org/}

\hypertarget{ref-Ren2017}{}
Ren, R., Liu, G., Wen, M., Horton, R., Li, B., \& Si, B. (2017). The
effects of probe misalignment on sap flux density measurements and in
situ probe spacing correction methods. \emph{Agricultural and Forest
Meteorology}, \emph{232}, 176--185.
\url{http://doi.org/10.1016/j.agrformet.2016.08.009}

\hypertarget{ref-Reyes-Acosta2012}{}
Reyes-Acosta, J. L., Vandegehuchte, M. W., Steppe, K., \& Lubczynski, M.
W. (2012). Novel, cyclic heat dissipation method for the correction of
natural temperature gradients in sap flow measurements. Part 2.
Laboratory validation. \emph{Tree Physiology}, \emph{32}(7), 913--929.
\url{http://doi.org/10.1093/treephys/tps042}

\hypertarget{ref-Rubilar2017}{}
Rubilar, R. A., Hubbard, R. M., Yañez, M. A., Medina, A. M., \&
Valenzuela, H. E. (2017). Quantifying differences in thermal dissipation
probe calibrations for Eucalyptus globulus species and E. nitens ×
globulus hybrid. \emph{Trees - Structure and Function}, \emph{31}(4),
1263--1270. \url{http://doi.org/10.1007/s00468-017-1545-3}

\hypertarget{ref-Sakuratani1981}{}
Sakuratani, T. (1981). A Heat Balance Method for Measuring Water Flux in
the Stem of Intact Plants. \emph{J. Agr. Met.}, \emph{37}(1964), 9--17.
\url{http://doi.org/10.2480/agrmet.37.9}

\hypertarget{ref-Schielzeth2013}{}
Schielzeth, H., \& Nakagawa, S. (2013). Nested by design: Model fitting
and interpretation in a mixed model era. \emph{Methods in Ecology and
Evolution}, \emph{4}(1), 14--24.
\url{http://doi.org/10.1111/j.2041-210x.2012.00251.x}

\hypertarget{ref-Schlesinger2014}{}
Schlesinger, W. H., \& Jasechko, S. (2014). Transpiration in the global
water cycle. \emph{Agricultural and Forest Meteorology}, \emph{189-190},
115--117. \url{http://doi.org/10.1016/j.agrformet.2014.01.011}

\hypertarget{ref-Schreel2018}{}
Schreel, J. D. M., \& Steppe, K. (2018). Analysis of sap flow dynamics
in saplings with mini-HFD (heat field deformation) sensors. In
\emph{Acta horticulturae} (pp. 161--166). International Society for
Horticultural Science (ISHS), Leuven, Belgium.
\url{http://doi.org/10.17660/ActaHortic.2018.1222.33}

\hypertarget{ref-Shimizu2015}{}
Shimizu, T., Kumagai, T., Kobayashi, M., Tamai, K., Iida, S., Kabeya,
N., \ldots{} Shimizu, A. (2015). Estimation of annual forest
evapotranspiration from a coniferous plantation watershed in Japan (2):
Comparison of eddy covariance, water budget and sap-flow plus
interception loss. \emph{Journal of Hydrology}, \emph{522}, 250--264.
\url{http://doi.org/10.1016/j.jhydrol.2014.12.021}

\hypertarget{ref-Simpson1993}{}
Simpson, W. T. (1993). Specific Gravity , Moisture Content , and Density
Relationship for Wood. \emph{Gen Tech Rep Fplgtr76 Madison WI US
Department of Agriculture Forest Service Forest Products Laboratory 13
P}, \emph{Gen. Tech.}, 13. Retrieved from
\href{http://citeseerx.ist.psu.edu/viewdoc/download?doi=10.1.1.155.4926\%7B/\&\%7Drep=rep1\%7B/\&\%7Dtype=pdf}{http://citeseerx.ist.psu.edu/viewdoc/download?doi=10.1.1.155.4926\{\textbackslash{}\&\}rep=rep1\{\textbackslash{}\&\}type=pdf}

\hypertarget{ref-Smith1995}{}
Smith, D. M. (1995). \emph{Water Use by Windbreak Trees in the Sahel}
(PhD thesis). University of Edinburgh.

\hypertarget{ref-Smith1996}{}
Smith, D., \& Allen, S. (1996). Measurement of sap flow in plant.pdf.

\hypertarget{ref-Sperling2012}{}
Sperling, O., Shapira, O., Cohen, S., Tripler, E., Schwartz, A., \&
Lazarovitch, N. (2012). Estimating sap flux densities in date palm trees
using the heat dissipation method and weighing lysimeters. \emph{Tree
Physiology}, \emph{32}(9), 1171--1178.
\url{http://doi.org/10.1093/treephys/tps070}

\hypertarget{ref-Steppe2010}{}
Steppe, K., De Pauw, D. J. W., Doody, T. M., \& Teskey, R. O. (2010). A
comparison of sap flux density using thermal dissipation, heat pulse
velocity and heat field deformation methods. \emph{Agricultural and
Forest Meteorology}, \emph{150}(7-8), 1046--1056.
\url{http://doi.org/10.1016/j.agrformet.2010.04.004}

\hypertarget{ref-Steppe2015}{}
Steppe, K., Vandegehuchte, M. W., Tognetti, R., \& Mencuccini, M.
(2015). Sap flow as a key trait in the understanding of plant hydraulic
functioning. \emph{Tree Physiology}, \emph{35}(4), 341--345.
\url{http://doi.org/10.1093/treephys/tpv033}

\hypertarget{ref-Suleiman1999}{}
Suleiman, B. M., Larfeldt, J., Leckner, B., \& Gustavssor, M. (1999).
Thermal conductivity and diffusivity of wood. \emph{Wood Science and
Technology}, \emph{33}(6), 465--473.
\url{http://doi.org/10.1007/s002260050130}

\hypertarget{ref-Sun2012}{}
Sun, H., Aubrey, D. P., \& Teskey, R. O. (2012). A simple calibration
improved the accuracy of the thermal dissipation technique for sap flow
measurements in juvenile trees of six species. \emph{Trees - Structure
and Function}, \emph{26}(2), 631--640.
\url{http://doi.org/10.1007/s00468-011-0631-1}

\hypertarget{ref-Swanson1983}{}
Swanson, R. (1983). \emph{Numerical and experimental analyses of
implanted probe heat pulse velocity theory} (PhD thesis). Univ. Alberta,
Edmonton, Canada.

\hypertarget{ref-Swanson1981}{}
Swanson, R., \& Whitfield, W. (1981). A numerical analysis of heat pulse
velocity theor. \emph{Journal of Experimental Botany}, \emph{32}(126),
221--239.

\hypertarget{ref-Taneda2008}{}
Taneda, H., \& Sperry, J. S. (2008). A case-study of water transport in
co-occurring ring- versus diffuse-porous trees: Contrasts in
water-status, conducting capacity, cavitation and vessel refilling.
\emph{Tree Physiology}, \emph{28}(11), 1641--1651.
\url{http://doi.org/10.1093/treephys/28.11.1641}

\hypertarget{ref-Testi2009}{}
Testi, L., \& Villalobos, F. J. (2009). New approach for measuring low
sap velocities in trees. \emph{Agricultural and Forest Meteorology},
\emph{149}(3-4), 730--734.
\url{http://doi.org/10.1016/j.agrformet.2008.10.015}

\hypertarget{ref-Uddling2009}{}
Uddling, J., Teclaw, R. M., Pregitzer, K. S., \& Ellsworth, D. S.
(2009). Leaf and canopy conductance in aspen and aspen-birch forests
under free-air enrichment of carbon dioxide and ozone. \emph{Tree
Physiology}, \emph{29}(11), 1367--1380.
\url{http://doi.org/10.1093/treephys/tpp070}

\hypertarget{ref-Vandegehuchte2012}{}
Vandegehuchte, M. W., \& Steppe, K. (2012a). A triple-probe heat-pulse
method for measurement of thermal diffusivity in trees.
\emph{Agricultural and Forest Meteorology}, \emph{160}, 90--99.
\url{http://doi.org/10.1016/j.agrformet.2012.03.006}

\hypertarget{ref-Vandegehuchte2012a}{}
Vandegehuchte, M. W., \& Steppe, K. (2012b). Interpreting the Heat Field
Deformation method: Erroneous use of thermal diffusivity and improved
correlation between temperature ratio and sap flux density.
\emph{Agricultural and Forest Meteorology}, \emph{162-163}, 91--97.
\url{http://doi.org/10.1016/j.agrformet.2012.04.013}

\hypertarget{ref-Vandegehuchte2012c}{}
Vandegehuchte, M. W., \& Steppe, K. (2012c). Sapflow+: A four-needle
heat-pulse sap flow sensor enabling nonempirical sap flux density and
water content measurements. \emph{New Phytologist}, \emph{196}(1),
306--317. \url{http://doi.org/10.1111/j.1469-8137.2012.04237.x}

\hypertarget{ref-Vandegehuchte2013}{}
Vandegehuchte, M. W., \& Steppe, K. (2013). Sap- fl ux density
measurement methods : working principles and applicability.
\emph{Fumctional Plant Biology}, 213--223.
\url{http://doi.org/http://dx.doi.org/10.1071/FP12233}

\hypertarget{ref-Vandegehuchte2015}{}
Vandegehuchte, M. W., Burgess, S. S., Downey, A., \& Steppe, K. (2015).
Influence of stem temperature changes on heat pulse sap flux density
measurements. \emph{Tree Physiology}, \emph{35}(4), 346--353.
\url{http://doi.org/10.1093/treephys/tpu068}

\hypertarget{ref-Vandegehuchte2012b}{}
Vandegehuchte, M. W., Steppe, K., \& Phillips, N. (2012). Improving sap
flux density measurements by correctly determining thermal diffusivity,
differentiating between bound and unbound water. \emph{Tree Physiology},
\emph{32}(7), 930--942. \url{http://doi.org/10.1093/treephys/tps034}

\hypertarget{ref-Vergeynst2014}{}
Vergeynst, L. L., Vandegehuchte, M. W., McGuire, M. A., Teskey, R. O.,
\& Steppe, K. (2014). Changes in stem water content influence sap flux
density measurements with thermal dissipation probes. \emph{Trees -
Structure and Function}, \emph{28}(3), 949--955.
\url{http://doi.org/10.1007/s00468-014-0989-y}

\hypertarget{ref-Wei2017}{}
Wei, Z., Yoshimura, K., Wang, L., Miralles, D. G., Jasechko, S., \& Lee,
X. (2017). Revisiting the contribution of transpiration to global
terrestrial evapotranspiration. \emph{Geophysical Research Letters},
\emph{44}(6), 2792--2801. \url{http://doi.org/10.1002/2016GL072235}

\hypertarget{ref-insidewood}{}
Wheeler, E. A. (2011). Inside Wood -- A Web resource for hardwood
anatomy. \emph{IAWA Journal}, \emph{32}(2), 199--211.
\url{http://doi.org/https://doi.org/10.1163/22941932-90000051}

\hypertarget{ref-Wiedemann2013}{}
Wiedemann, A., Jiménez, S. M., Rebman, C., Cuntz, M., \& Herbst, M.
(2013). Empirical study of wound response dynamics on sap flow measured
with thermal dissipation probes. International Society for Horticultural
Science.

\hypertarget{ref-Wiedemann2016}{}
Wiedemann, A., Marañón-Jiménez, S., Rebmann, C., Herbst, M., Cuntz, M.,
Walther, B. A., \ldots{} Rahbek, C. (2016). An empirical study of the
wound effect on sap flux density measured with thermal dissipation
probes. \emph{Tree Physiology}, \emph{36}(12), 1471--1484.
\url{http://doi.org/10.1093/treephys/tpw071}

\hypertarget{ref-Wilson2001}{}
Wilson, K. B., Hanson, P. J., Mulholland, P. J., Baldocchi, D. D., \&
Wullschleger, S. D. (2001). A comparison of methods for determining
forest evapotranspiration and its components: Sap-flow, soil water
budget, eddy covariance and catchment water balance. \emph{Agricultural
and Forest Meteorology}, \emph{106}(2), 153--168.
\url{http://doi.org/10.1016/S0168-1923(00)00199-4}

\hypertarget{ref-Wullschleger2011}{}
Wullschleger, S. D., Childs, K. W., King, A. W., \& Hanson, P. J.
(2011). A model of heat transfer in sapwood and implications for sap
flux density measurements using thermal dissipation probes. \emph{Tree
Physiology}, \emph{31}(6), 669--679.
\url{http://doi.org/10.1093/treephys/tpr051}

\hypertarget{ref-Wullschleger1998}{}
Wullschleger, S. D., Meinzer, F. C., \& Vertessy, R. A. (1998). A review
of whole-plant water use studies in tree. \emph{Tree Physiology},
\emph{18}(8-9), 499--512.
\url{http://doi.org/10.1093/treephys/18.8-9.499}

\hypertarget{ref-Xie2018}{}
Xie, J., \& Wan, X. (2018). The accuracy of the thermal dissipation
technique for estimating sap flow is affected by the radial distribution
of conduit diameter and density. \emph{Acta Physiologiae Plantarum}.
\url{http://doi.org/10.1007/s11738-018-2659-y}

\hypertarget{ref-Zanne2010}{}
Zanne, A. E., Westoby, M., Falster, D. S., Ackerly, D. D., Loarie, S.
R., Arnold, S. E., \& Coomes, D. A. (2010). Angiosperm wood structure:
Global patterns in vessel anatomy and their relation to wood density and
potential conductivity. \emph{American Journal of Botany}, \emph{97}(2),
207--215. \url{http://doi.org/10.3732/ajb.0900178}

\hypertarget{ref-Zhang2014}{}
Zhang, Q., Manzoni, S., Katul, G., Porporato, A., \& Yang, D. (2014).
The hysteretic evapotranspiration---Vapor pressure deficit relation.
\emph{Journal of Geophysical Research: Biogeosciences}, \emph{119},
125--140. \url{http://doi.org/10.1002/2013JG002484.Received}


% Index?

\end{document}
